\documentclass[11pt, english]{article}
        \usepackage{geometry}
                \geometry{
                        a4paper,total={210mm,297mm},
                        tmargin=40.8mm,
                        bmargin=40.8mm,
                        lmargin=32.6mm,
                        rmargin=32.6mm,
                }

        \usepackage{titlesec}
                \titleformat{\section}
                        {\normalfont\fontsize{18}{16}\bfseries}{\thesection}{0.5em}{}
                \titleformat{\subsection}
                        {\normalfont\fontsize{14}{16}\bfseries}{\thesubsection}{1em}{}
                \titleformat{\subsubsection}
                        {\normalfont\fontsize{11}{16}\bfseries}{\thesubsubsection}{1em}{}

        \usepackage{longtable}
        \usepackage{multirow}

        \usepackage[labelfont=bf,textfont=bf,font=small,skip=8pt]{caption}
        
        \usepackage{hyperref}
                \hypersetup{
                        colorlinks=true,
                        linkcolor=black,
                        filecolor=magenta,
                        urlcolor=cyan,
                        }

        \setlength{\parindent}{0pt}
        \renewcommand{\baselinestretch}{1.25}
        \usepackage{setspace}

	\usepackage{multirow}

\begin{document}

\pagenumbering{gobble}

	\title{\textsc{AG431 Corporate Investment\\ Coursework Assignment}}
        \author{\textsc{Lewis Britton 201724452}}
        \date{\textsc{Academic Year 2020/2021}}
        \maketitle

        \begin{abstract}
		This work explores the merger of Standard Life PLC and Aberdeen Asset Management PLC, to form Standard Life Aberdeen PLC. Examining merger mechanics, economies of relevant scale and scope, merger wave trends, and proposed synergies; it finds some success. The merger has fairly successfully seen cost-based synergy improvements, through reduced relevant operational costs, in \textit{Operating Platforms}, \textit{Distribution Overlap}, \textit{Central Functions} and \textit{Pool of Investment}. These immediately apparent areas for enhanced efficiency have yet to be aided by other proposed revenue-based synergies such as improvement of \textit{Global Market Access}, for example. Additionally, factors such as AUM decline, annual profit decline, large implementation costs etc., highlight areas of failure however.
        \end{abstract}

	\textbf{Index terms:} Merger, Horizontal, Congeneric, Standard Life PLC (SL), Aberdeen Asset Management PLC (AAM), Standard Life Aberdeen PLC (SLA), Assets Under Management (AUM), Investment Management, Pension \& Savings, International Business, Corporate Relationships, Operating Platforms, Distribution, Central Functions, Complementary Capabilities, Global Market Access, Pool of Investment, Efficiency Theory, Synergy Theory, Diversification Theory, Strategic Realignment Theory, Market Powder Theory

\newpage

\pagenumbering{roman}

	\renewcommand{\contentsname}{Table of Contents}

        \tableofcontents

\newpage

\section*{Question Summary}

	\begin{enumerate}
		\item What type of merger was it? Discuss the potential for economies of scale and scope.
		\item Would you consider it to have been part of an industry merger wave?
		\item Discuss the merger synergies outlined in both the prospectus and the scheme                     
document in light of class discussions.
\item Based on the evidence thus far would you consider the merger to have been a success or failure. Try to relate your answer to the merger theories discussed in the course. You may wish to focus on merger integration, particularly in relation to Sections 7 and 8 of the Prospectus and Sections 5-7 of the Scheme document.
	\end{enumerate}

	\textit{For the final question you will need to download the latest balance sheet and/or look at recent articles on Aberdeen Standard Life in the financial press (please provide appropriate attribution)}

\newpage

\pagenumbering{arabic}

\section{Introduction}

The adjoining of Standard Life PLC (SL) and Aberdeen Asset Management PLC (AAM) of March 2017 formed Standard Life Aberdeen PLC (SLA). It came during an M\&A surge cooldown; in a year reflecting \~{}\$3.776tn worth of M\&A (Statista, 2021 [2]). The value of this deal was \pounds3.2bn paid by acquirer, SL (Table 1). The newly formed company became the largest UK asset manager at the time; ranked the third largest by 25.8\% (ADV Ratings, 2021), using an Assets Under Management (AUM) metric. AUM of SLA is slightly lesser than the sum of both firms pre-merger (Standard Life Aberdeen, 2019); partially offset by a large 2018 legal settlement. The merger was designed to allow SL to optimise cross-overs through \textit{synergies} and {economies of scale/scope} increasing global real estate, and halting unwanted over-concentration in insurance\footnote{This was reflected in SL's subsequent sale of their insurance business to Phoenix Group}.

\newpage

\section{Merger Type}

	\subsection{Merger Mechanics}

Mergers such as this call for mutuality of interests, growth aspirations, and efficiency-enhancing concepts (Cheng, et al. 2007). This was not a hostile takeover. It was a horizontal, congeneric merger between similar firms. SL's 2016 half-year report states AUM of \pounds328.0bn (Standard Life PLC, 2016). AAM's; a sum of of \pounds312.1bn (Aberdeen Asset Management PLC, 2016). Thus, the firms were operating [approximately equivalently in respective hierarchies] in overlapping business such as insurance, pensions; with promise for scale enhancement. Also, with [complimentary-extension opportunities] in differing business such as private equities and global operations. (Tables 5 \& 6). Thus, horizontal-congeneric (Table 7). This supports mutual interest in expanding client base and using shared resources/methodologies as efficiency stimulant. The merger seeks improved choice, prolonged investment, expanded specializations/brands/networks, talent acquisition, greater dividends and, optimized profits/costs (Table 2).

	\subsection{Economies of Scale \& Scope}

Firms achieve positive economies of scale when joint efforts increase efficiency; optimizing profits/costs. SLA, wish to reduce market competition using \textit{Market Power Theory} (Weston, et. al, 2010)\footnote{More in Table 10}. Using enhanced scale and of combined volume allows ever-increasing insurance service, pension/savings demand to be met (Table 3). Scope of incorporating non-crossover operations; creates new international business networks and corporate relationships, also at a cheaper cost than when operating individually (Table 3).\\

The letter from the CEO (Aberdeen Standard Investments, 2017) proposes `1000 investment professionals', `24 investment centres', and `20 countries'. This highlights enhanced scale matching employees with greater focus on existing specializations, and scope matching employees/locations with new/expanded networks/foreign networks and more efficient operation re-alignment. Thus, offering greater versions of an increased variety of services (Table 3). This delivers enhanced quality, client reassurance, improved relationships, more efficient cash purposing (in cross-over/work-force/supply/infrastructure etc.), less spent on competitive, stronger legal standing, stronger growth etc.

\newpage

\section{Industrial Waves}

	\subsection{Industrial Context}

The year of this merger a reflected \~{}\$3.776tn worth of M\&A (Statista, 2021 [2]). 2017 was the second-greatest year for financial service M\&A (Statista, 2021 [1]). It could be argued the SLA merger was part of a `wave'. Merger waves are based upon common frameworks/mechanics of firms for more macroscopic reasons. The \pounds3.2bn value of the SL/AAM merger was global-top-tier however, merger activity was down by \~{}35\% in 2017, financial services by 5.4\%, (NASDAQ, 2017). The overwhelming 2017 wave was technology-oriented; based on huge industry-level NASDAQ activity. It saw Amazon's world-famous acquisition of Whole Foods, and further' by Intel, Cisco, Apple, Alphabet. Financial services saw their second-highest year (\pounds2.119tn) of M\&A meaning SLA may have carried wave characteristics.

	\subsection{Relevant Indicators}

Much like waves of 1922--1929 and 1981--1989, there were various industrial legal/regu-latory drivers in 2017, alongside moderate economic growth (Table 9). 2017 contained low volatility trends, as little-as-so as the early 1900's period, and technological return was up by \~{}38\% (CNBC, 2017). 2017 also saw regulation tightening around disclosure of asset management firms' fees/liquidity, for a more accurate view of AUM etc. (The Economist, 2017); also further regulation upon relevance of client asset advisory. Thus, this period saw clients moving to online asset management services, not associated with `traditionals' SL/AAM.\\

There wasn't an industry wave. However, the circumstances created an industrial issue suspending traditional scope. Thus, to attain goals such as improved client choice (reacting to perhaps lower-value assets they're require to recommend to client needs), prolonged active investment (reputation), global demand (without going fully digital), maintaining brands and networks (Tables 2 \& 3); mergers make sense. Especially when discarding unoptimized management/workforce. SLA could use their new scope to increase AUM volume while accepting/adopting technological changes/preferences. Thus, generating brand-new revenue greater than possible losses from shifting regulation and technological preferences. This is particularly relevant to strategies of SLA such as international business as they wish to maintain relationships/reputation with the global networks of both SL and AAM as they expand (Table 4).

\newpage

\section{Merger Synergies}

	\subsection{Merger Synergy Outline}

\textit{Synergy Gain Theory}\footnote{This study makes use of various merger theories, see Table 10 for more} refers to when two firm's sum output value is greater than the whole output value (Romano, 1992). Hence, efficiency stimulus. Synergies focus on cost-revenue optimization, making use of economies of scale/scope. Synergies should utilise inventory/supply/service/manu-facturing to produce greater volume for lower cost-per-[unit/service] (scale) and, fill non-crossovers in adjacent firms to widen reach offerings (scope).

	\subsection{Proposed Enhancements}

SLA propose four cost-reduction synergies (Table 4), forecasted to generate \pounds200m pre-tax cost reductions p.a. by year-three end post-merger. [1] overlapping operations: client/command-chain interfacing/communication platforms. \textit{Efficiency Theory} (Wolfe, et al, 2011) sees under-optimisation, associated fixed/floating costs, reduced. [2] overlapping distribution: networks/rel-ationships. \textit{Strategic Realignment Theory} (Westion, et. al, 2010) sees this removed and harmony realised. Time and costs are reduced and cash can be repurposed upon expanding \textit{scope} of distribution rather than wasted on `empty-overlap'. This is particularly apparent where the firms shared geographically similar logistic strategies. [3] overlapping central functions: workforces/specializations. Harmonising these removes cross-over real estate/operations/workforces. Further, centralizing \textit{scope} of specialization networks for enhanced \textit{scale} through time-efficiency and reduced talent costs. [4] the formers imply additional cost-savings on real estate, management fees, travel (and travel time), legalities, professional/consulting. This is again enhanced by \textit{Efficiency Theory}.\\

Four revenue-increasing synergies are proposed (Table 4) as a result of complimentary investment capabilities, client footprints and distribution relationships. [1] customer franchise : expanded \textit{scope} of client type and \textit{scale} of clients-service-specialzation. \textit{Efficiency/Diversification Theory} see merged firms providing greater-quality service to existing clients and attracting broader range through greater capabilities. [2] complementary investment: alignment of greater \textit{scale/scale} of offering due to higher volume of client interest who arrive from one area and find others. These should also be better-optimized with combined specialization. \textit{Diversification Theory} extends this into AAM's Asian networks. [3] global access: joining of domestic/foreign geographical infrastructure offers greater UK \textit{scale}, where both firms exceed; and, greater \textit{scope} (\textit{Diversification Theory}) in Asia, North/South America where the adjacent firm has smaller/non-existent operations. [4] investment pool: \textit{scope} of interest (clients/investors) is widened as not only does the merged firm offer greater variety, offerings should also be of greater caliber due to complements/alignments. This aligns with \textit{Efficiency/Diversification Theory}.

\newpage

\section{Merger Success}

Pre-merger AUM totalled \pounds640.1bn, with 2019's at \pounds544.6bn. This doesn't satisfy \textit{Synergy Gain} merger success; indicating failure. This is seconded by the 62.9\% drop in annual profit from 2017--2019. However, the issue is more complex as initial stages are primarily cost-based. In 2016, SL reported \pounds328.0 of AUM. This was based on investment management, pensions/savings and Indian/Chinese operations. Foreign revenue was decreasing from 2014. As of 2016, AAM reported AUM of \pounds312.1. 5,400 of 6,302 of SL's staff were located in the UK, with the majority of their AUM also appearing in the UK (Table 5). \~{}64\% of the staff were focused on pension and savings; this \textit{scope} should be enhanced by realignment of staff and \textit{scale} from AAM. \~{}60\% of AAM's business was UK-based. As of 2019\footnote{Source: Standard Life Aberdeen PLC Financial Report 2019}. Institutional/wholesale fee-based revenue prevailed at \pounds1011m, next to insurance at \pounds317m.\\

SLA extended synergy integration until 2021 due to `technological/infrastructural issues'. They claim \pounds400m of synergy realization by 2021's end. Synergies realized to 2019 were valued at \pounds114m. These were primarily cost-based based, on rationalisation of premises and workforce/supplier efficiencies, reducing staff costs, and reducing \textit{Operating Platforms}. This aided further \pounds62m of synergies in \textit{Distribution Overlap}; furthered by \textit{Pool of Investment} reducing financing costs associated with acquiring clients/talent. 2019 didn't focus on revenue-enhancements such as \textit{Global Market Access}. Despite positive synergies, implementation costs summed \pounds436m, with an estimated total of \pounds555m($\uparrow$\pounds125m). However, 2019 saw a 4\%($\downarrow$\pounds62m) reduction in operating expenses to \pounds1,333m\footnote{Accounting for costs of inflation, partnerships/acquisitions and foreign exchange}. Although implementation costs are far greater than realized synergies, SLA made reasonable progress towards their \pounds200myear$^{-1}$ synergy realization.\\

By 2020\footnote{Source: Standard Life Aberdeen PLC Half-Term Financial Report 2020} half-term, an additional \pounds34m of synergies were realized from staff optimization and logistical synchronisation (\textit{Operating Platforms}). A further \pounds57m was realized from localizing infrastructure and simplifying supply networks (\textit{Central Functions}). This focuses again on cost-refinement, not revenue-enhancement; neglecting scope from \textit{Global Market Access} and \textit{Client Opportunities}. Updated 2020 implementation costs are \pounds482m($\uparrow$\pounds44m) so centralizations are assisting. Total realized synergies are \pounds323m, with a greater \pounds137m reduction in operating expenses.\\

SLA are on track for \pounds200myear$^{-1}$ cost-based synergy realization but, not all grounds have been coverd. Current priorities/refinements do however, aid future revenue-based enhancements. Implementation costs have been fairly high, remaining at total expected \pounds555m. The synergy goal remains at \pounds350m (2020) plus \pounds50m (2021). Dividends are to remain constant at 21.6pshare$^{-1}$ until focus on revenue-based synergy, when they're proposed to decrease to 14.3pshare$^{-1}$. Here, much of the \pounds3.7bn cash/liquidity and dividend savings will be focused on further (growth/scope) synergies and maintaining a consistent dividend (reliability), in-line with remaining rationale (Table 2). This merger has seen successful cost-mimizations however, failure of revenue-maximisation, particularly in their \pounds109bn loss of business from Lloyds Banking Group, for example. There is neglect of \textit{International Business} and \textit{Corporate Relationships}

\newpage

%\textsc{Won't Be Included In Text, Just For Reference}\\
%
%\textit{Changing Firm Value}
%
%\begin{itemize}                 
%\setlength\itemsep{0cm}
%	\item $R_t$: revenue 
%	\item $g$: growth rate
%	\item $m$: net operating income margin (revenue $-$ COGS)
%	\item $T$: actual tax rate
%	\item $I_t$: investment (working cap $+$ fixed assets) as ratio of revenue
%	\item $n$: number of periods of supernormal growth
%	\item $k$: cost of capital
%\end{itemize}
%
%Value w/ Constant Growth:
%
%$$\mathrm{V_0=\frac{R_0(1+g)\left(m(1-T)-I\right)}{k-g}}$$
%
%Value w/ Temporary Supernormal Growth \& Constant Growth:
%
%$$\mathrm{V_0=R_0\left(m(1_T)-I_s\right)\sum_{t=1}^n\frac{(1+g_s)^t}{(1+k)^t}+\frac{R_0(1+g_s)^n\left(m(1-T)-I_c\right)}{(1+k)^n}\left(\frac{(1+g_c)}{(k-g_c)}\right)}$$
%
%\textit{Refer To}

%\begin{tabular}{p{5cm}p{7cm}}
%
%\begin{itemize}
%\setlength\itemsep{0cm}
%	\item Prospectus: 7 \& 8
%	\item Scheme: 5, 6 \& 7
%\end{itemize}
%
%	&
%
%\begin{itemize}
%\setlength\itemsep{0cm}
%	\item Efficiency Theory 
%	\item Synergy Theory 
%	\item Diversification Theory 
%	\item Strategic Realignment Theory
%	\item Market Power Theory
%\end{itemize}
%
%\end{tabular}
%
%\newpage
%
\pagenumbering{Roman}

\section*{Appendices}

	\subsection*{Appendix 1: Merger Information}

\begin{table}[h]
        \scriptsize
	\renewcommand{\arraystretch}{1.25}
\begin{center}
\begin{tabular}{ p{5cm} p{7.5cm} }
        \hline
        \textbf{Adjoining Firms} & Standard Life PLC \& Aberdeen Asset Management PLC\\
        \textbf{Acquirer} & Standard Life PLC\\
        \textbf{Formed Firm} & Standard Life Aberdeen PLC\\
        \textbf{Paid By Acquirer} & \pounds3.2bn\\ 
        \textbf{Ownership Share} & $0.\dot{6}$ (SL); $0.\dot{3}$ (AAM)\\
        \textbf{Share Issuance} & 999,848,295\\
        \textbf{Share Value} & 12$\frac{2}{9}$GBX\\
        \textbf{SL$^1$ Pre-Merger AUM$^2$} & \pounds328.0bn\\
        \textbf{AAM$^3$ Pre-Merger AUM} & \pounds312.1bn\\
	\textbf{SLA$^4$ Post-Merger AUM (2019)} & \pounds544.6bn\\
        \hline
        \multicolumn{2}{p{12cm}}{$^1$: Standard Life PLC}\\
        \multicolumn{2}{p{12cm}}{$^2$: Assets Under Management}\\
        \multicolumn{2}{p{12cm}}{$^3$: Aberdeen Asset Management PLC}\\
        \multicolumn{2}{p{12cm}}{$^4$: Standard Life Aberdeen PLC}\\
        \hline
\end{tabular}
        \caption{Basic Merger Details}
\end{center}
\end{table}
%
%\newpage
%
%\begin{table}[h]
%        \footnotesize
%\begin{center}
%\begin{tabular}{p{5cm}p{6cm}}
%        \textbf{Date} & \textbf{Event}\\
%        \hline
%        09 May 2017 & Prospectus, Circular and Scheme Document publication\\
%        15 June 2017 & Voting record time for Standard Life General Meeting\\
%        15 June 2017 & Voting record time for Aberdeen Asset Management Meetings\\
%        19 June 2017 & Aberdeen Court Meeting\\
%        19 June 2017 & Aberdeen General Meeting\\
%        19 June 2017 & Standard Life General Meeting\\
%        11 August 2017 & Aberdeen Court Hearing\\
%        11 August 2017 & Last day for dealings in, for registration of transfers of and, disablement in CREST of, Aberdeen Shares\\
%        11 August 2017 & Suspension of listing of and, dealings in, Aberdeen Shares\\
%        14 August 2017 & Effective Date\\
%        14 August 2017 & Admission of dealings in New Shares on London Stock Exchange\\
%        14 August 2017 & New Shares issued and credited to CREST accounts\\
%        14 August 2017 & Delisting of Aberdeen Asset Management Shares\\
%        Within 14 Days of Effective Date & Latest date for [1] CREST accounts to be credited in respect of New Shares and assured payment obligations in respect of any cash due and [2] dispatch of share certificates in respect of the New Shares and cheques in respect of fractional entitlements to New Shares\\
%        31 December 2017 & Long Stop Date\\
%        \hline
%        \multicolumn{2}{l}{\textit{Source: Prospectus \& Scheme Document}}\\
%        \hline
%\end{tabular}
%\end{center}
%        \caption{Standard Life Aberdeen Merger Timeline}
%\end{table}

\newpage

\begin{center}
        \scriptsize
\begin{longtable}{p{4cm}p{8.5cm}}
        \textbf{Reason} & \textbf{Rationale}\\
        \hline  
        Improved Choice & Both firms operate very similarly. They both dominate the market and offer wide portfolios. Joining the firms means that their respective portfolios will become more diversified and the skills of employees will be complimentary due to the harmony in operations. Operations extend to developed and developing economic markets, portfolio management, other-asset management, real estate etc. Hence, a wide offering to combined clients. There is also a claim of low overlap within specific asset operations, meaning joining efforts will not mean the cancellation of much business; business simply multiplies.\\
        Prolonged Active Investment & Both firms claim to seek a long-lasting reputation in anaging assets, i.e. they do not wish to transition operations to other sectors/subsectors as they hold strong reputation and performance in their current business. The similar strategies and operations but new combination of talent are said to work in harmony to refresh general corporate quality, client offerings offerings, diversity etc. In result, prolonging the life of the newly formed firm.\\
        Demand Requirements & As both firms are already very large industry leaders, joining forces only enhances this. The combination of talents should theoretically exponentially grow the firm meaning they can acquire more global clients outwith their current ones. It's believed that the new firms are able to adapt to changing global economies more efficiently and on a larger scale, therefore having the knowledge, skills and ability to take on new types of (perhaps larger) clients. This larger scale also should be able to produce better innovation and development. Reflecting this, it's claimed that the new firm's insurance assets are forecast to grow by \$750bn in 2015--2020. This is due to both fimrs already holding a huge footprint in these areas. Its also claimed that much of the innovation will take place within offering new, global, methods in managing the changing needs of individual's savings and retirement funds. Standard Life saw success in this area and believe this will be the next big area to grow post-merger.\\
        Brands \& Networks & Both firms own reputable brands and chains of operations such as client-scope-enhancing methods such as specific client seeking and marketing etc. This is as simple as the fact that when two reputable set of brands join, they tend to make equally, if not more, reputable and stable brands in the future. Therefore, leading to further improved networs, reputation and client relationships. The geographical expansion is said to hold 50 locations of operation; Aberdeen AM aid the existing distribution by offering central Asian and US location and relationships. This of course, is in search of a further diversified client network and scope.\\
        Talent Acquisition & With the enhanced scale and scope of reputation, networks and assets under management etc.; and, not forgetting the fact that the new firm is one of the largest of its type; greater volumes of higher-quality talent should be attracted from corporate, employee and client perspectives. In theory, this means further enhanced efficiency and investment. This is in hopes of increasing fixed/long-term realtionships and income. Talented and skilled people attract good business. This also holds promise for new perspectives and opinions on development in technological, logistical and risk areas.\\
	Specific Specializations & As discussed, Standard Life are leaders in pension management. Therefore, not only does the merger aid in developing business operation scope on a whole, it also gives a new persepctive on the enhanacements of existing specializations such as this. The same argument holds for the previously discussed strength of insurance asset management, of both firms.\\
        Dividends & A goal for the new firm is to generate the appropraite revenue from diversification to offer attractive long-term returns and dividends to their shareholders. That is, maintaining and growing their reputation and relationship with existing and new shareholders. Standard Life had a strong pre-merger dividend policy. They wish to use the joint resources to enhance this.\\
        Profit \& Cost Benefits & It's estimated that the synergies proposed in this merger will generate cash savings of $\pounds$200m by the end of year-three succeeding the merger. 75\% of this is said to be accomplished by the end of year-two succeeding the merger. Along with the discussed revenue generation, the proportionately imporved cost savings and profits are projected to be large.\\ 
        \hline
        \multicolumn{2}{l}{\textit{Source: Prospectus Scheme Document}}\\
        \hline
        \caption{Merger Rationale}
\end{longtable}
\end{center}

\newpage

\begin{center}
        \scriptsize
\begin{longtable}{p{4cm}p{8.5cm}}
        \hline
        \multicolumn{2}{p{13cm}}{The newly formed company aim to use their discussed rationale to further grow, specialise in, and maximise value of the following areas of operations; with the aim of being one of, if not the, best offerer of these in the UK and perhaps globally.}\\
        \hline
        \textbf{Operation} & \textbf{Strategy}\\
        \hline
        Investment Management & Primary operations remain in asset management. As discussed, firms merged with the aim to dominate this market and grow their reputation as a whole. A market-leading scale of investment opportunities and funds are made available. Services are provided to clients through institutional and wholesale methods.\\
        Pensions \& Savings & One of the largest operations in Standard Life. Together, pension and savings operations are market-dominating and extend to the UK, Ireland and Germany. Included in this branch is an array of pension, savings and other benefit schemes. They're operated through a chain of intermediaries such as financial advisors and direct contact with the client.\\
        International Business & As discussed, Aberdeen AM provides much Asian scope to operations. Imporved reach comes in places such as India, China and Hong Kong. its claimed that in India, the new firm could expand to 25m individual customers across operations. On the whole, the new firm's stake in large Indian asset and insurance comapies provides a strong advantage in growth rate. This is reflected in Standard Life's pre-merger 40\% holding in HDFC Asset Management Company Limited and 35\% holding in HDFC Standard Life Insurance Company Limited subsidiaries, in India. The new firm also shows a similar position in the Chinese and Hong Kong markets. A 50\% holding is present in Heng An Standard Life Insurance Company Limited, in China.\\
        Corporate Relationships & With the connections of SL and AAM, the network of connections available globally is proposed to be huge. Providing the merger does not cause any corporate changes of opinion. The biggest operational connections include Lloyds Banking Group, the largest UK commercial bank; Mitsubishi UFJ Financial Group, one of the largest commercial banks in Japan (Japan has a huge retirement market, aligning with the specialties of the merged firm); Phoenix Group, a large life insurance firm who's assets in which SL already have a steak and; various other US, Canadaian, Asian and Australasian firms. With this network, there is hige opportunity for the new firm to continue to communicate with and gain exposure to new and greater clients.\\
        \hline
        \multicolumn{2}{l}{\textit{Source: Prospectus \& Scheme Document}}\\
        \hline
        \caption{Strategies}
\end{longtable}
\end{center}

\newpage

\begin{center}               
        \scriptsize
\begin{longtable}{p{4cm}p{8.5cm}}
        \textbf{Synergy} & \textbf{Value Creation}\\
        \hline
        \multicolumn{2}{c}{\textbf{Cost-Based}}\\
        \multicolumn{2}{c}{\textit{Forecast \pounds200m Pre-Tax Cost Synergies Value p.a. By Yr.3 End}}\\
        \hline
        Operating Platforms (31\%) & Synchronised operations on equivalent and platforms simply save costs which are historically found in the individual under-optimisation or lack of ability to optimise resources and platforms as single firms. Essentially `filling in the gaps' of one-another to aid optimisation. Number of platforms used is also reduced, reducing fixed and some floating costs and; there is a reduction in the reliance on third parties meaning more efficient platforms arent only used for greater versions of already existing work, they're used for the increased scope of tasks too.\\
        Distribution Overlap (16\%) & Standard Life and Aberdeen AM as two entities have overlapping distribution networks. This is good in this case as the overlaps are removed upon merger meaning that the large spread of distribution represented by the sum of the two firms is concentrated; with focus and efforts also being concentrated in this regard. This means, time and money is saved in producing output. Again, the sum of the two firms efforts in this regard would be far greater then the merged efforts with the overlap removed. This is especially apparent where the individual firms use geographically close/similar logistic strategies, making the transition easier, also.\\
        Central Functions (12\%) & Much like the former, operations are also centralised, with overlapping or no longer necessary functions being removed or minimised. This way, operations, wokforce, efforts, finance, etc., is concentrated in the relevant areas and optimised as well as possible. Like the previous example; this optimisation is incredibly more efficient than each firm acting as a single entity.\\
        Other Costs \& Fees & It's claimed that savings come in the form of real estate, management fees, travel costs, legal fees and, professional/consulting fees. This is clearly evident as operations are more centralised and optimised. Naturally, a narrower spread/scope of real estate, management, etc., are required. Thus, again optimising concentration where relevant. Essentially, all `duplicates'/overlap is removed.\\
        \hline
        \multicolumn{2}{c}{\textbf{Revenue-Based}}\\
        \multicolumn{2}{c}{\textit{Result of ``Complementary Investment Capabilities, Client Footprints \& Distribution Relationships''}}\\
        \hline
        Customer Franchises & A wider scope and range of type of client and idvidual is made apparent by the joining of the firms. This means more opportunity for selling innovative and new services etc.\\
	Leveraging Complimentary Investment Capabilities & The firms have areas in which gaps can be filled by the adjacent firm. That is, something one of them does extremely well may compliment a need of a client of the other firm. This specific type of client may not be immediately apparent in the former firm's selection of clients therefore, they have been underoptimising this specialisation. The examples given are that Standard Life's multi-asset and risk management will find its fundamental value in Aberdeen AM's client base and, Aberdeen AM's emerging Asian market experience and capabilities will find its fundamental value in the growth of Standard Life.\\
        Global Market Access & Simply, scope and reach of markets is significantly increased over the joining of the firms. Infrastructure and operations appear now in many major Asian, North American, South American and UK locations. As two firms operating in their respective shares of those contitnents, they may not have held resources to obtain the optimal reach however, with the efforts of the joined firms, resources can be better used for this international growth and reach purpose.\\
        Pool of Investment & Investors, influencial forces and interest is widened by the joining of the firms. this means the opportunity of attracting development and innovation driven forces is more likely due to the increased chances of the correct aligning of intellect and interest.\\
        \hline
        \multicolumn{2}{c}{\textbf{Additional}}\\
        \hline
        \multicolumn{2}{p{13cm}}{In addition to cost-saving and profit-maximising synergies based around economies of scale and scope, the merging of the firms shows potential for `capital synergies' also. This, for example, includes reduction and concentration of legal and regulatory efforts, fees and outsourcing and; other chains/networks which act alike.}\\
        \hline
        \multicolumn{2}{l}{\textit{Source: Prospectus \& Scheme Document}}\\
        \hline
        \caption{Synergies}
\end{longtable}               
\end{center}

\newpage

\begin{center}
	\scriptsize
\begin{longtable}{p{5cm}p{6cm}}
	\hline
	\hline
	\multicolumn{2}{c}{\textbf{Insurance Business}}\\
	\hline
	\hline
	\textbf{Distribution} & \textbf{Products}\\
	\hline
	Institutional & Equities\\
	Wholesale & Fixed Income\\
	& Real Estate\\
	& Multi-Asset\\
	& Private Equities\\
	& Insurance\\
	\hline
	\multicolumn{2}{c}{\textbf{Markets}}\\
	\hline
	\textbf{Client Geography} & \textbf{2016 AUM (bn\pounds)}\\
	\hline
	UK & 144.4\\
	Europe & 16.2\\
	North America & 12.7\\
	Asia Pacific & 3.8\\
	India & 10.6\\
	\hline
	$\Sigma$ & 187.7\\
	\hline
	\hline
	\multicolumn{2}{c}{\textbf{Pension \& Savings Business}}\\
	\hline
	\hline
	\textbf{Distribution} & \textbf{Products}\\
        \hline
	Retail & Fee-Based\\
	Workplace & Spread \& Risk\\
	\hline
        \multicolumn{2}{c}{\textbf{Markets}}\\
        \hline
	\textbf{Client Geography} & \textbf{2016 PBT (P\&S) (m\pounds)}\\
	\hline
	UK & 319\\
	R.o. Ireland & \multirow{2}{*}{43}\\
	Germany & \\
	\hline
        \hline
	\multicolumn{2}{c}{\textbf{Indian \& Chinese Business}}\\
        \hline
        \hline
	\textbf{Distribution} & \textbf{Products}\\
        \hline
	Bancassurance (HDFC Life) & Individual Life Insurance\\
	& Group Life Insurance\\
	Banks \& Brokers (HASL) & Individual Health Insurance\\
	Direct (HASL) & Group Health Insurance\\
	\hline
        \textbf{Client Geography} & \textbf{2016 PBT (P\&S) (m\pounds)}\\         
        \hline
	India & 34\\
        China & 7\\
        Hong Kong & -5\\
	\hline
	\hline
	\multicolumn{2}{c}{\textbf{Strategic Framework}}\\
	\hline
	\hline
	\multicolumn{2}{p{12cm}}{\textit{Broadening and deepening investment capabilities $\rightarrow$ Building and efficient and effective business $\rightarrow$ Attracting, retaining and developing talented people $\rightarrow$ Growing and diversifying revenue and profit $\rightarrow$ Developing strong relationships with cutomers and clients}}\\
	\hline
        \hline
	\multicolumn{2}{c}{\textbf{Employees}}\\ 
        \hline
        \hline
	\textbf{Employee Geography} & \textbf{2016 N Employees}\\
	\hline
	Asia & 150\\
	Australia & 9\\
	Europe & 641\\
	North America & 102\\
	United Kingdom & 5400\\
	\hline
	$\Sigma$ & 6302\\
	\hline
	\textbf{Employee Segment} & \textbf{2016 N Employees}\\
	\hline
	Standard Life Investments & 1681\\
	Pensions \& Savings & 4026\\
	India \& China & 112\\
	Other & 483\\
	Canada & 0\\
	\hline
	$\Sigma$ & 6302\\
	\hline
	\multicolumn{2}{l}{\textit{Source: Prospectus \& Scheme Document}}\\
	\hline
	\caption{Standard Life PLC Information}
\end{longtable}
\end{center}

\newpage

\begin{center}
	\scriptsize
\begin{longtable}{p{5cm}p{6cm}}
	\hline
	\hline
	\multicolumn{2}{c}{\textbf{Markets (Third Party)}}\\
        \hline
	\hline
        \textbf{Client Geography} & \textbf{2016 Share of AUM}\\
	\hline
	UK & 57.4\%\\
	Europe & 18.5\%\\
	Asia & 5.9\%\\
	Americas & 16.3\%\\
	Middle East \& Africa & 1.9\%\\
	\hline
	$\Sigma$ (bn\pounds) & 324.4\\
	\hline
	\hline
	\multicolumn{2}{c}{\textbf{Strategic Framework}}\\
        \hline
	\hline
	\multicolumn{2}{p{12cm}}{\textit{Enhance capabilities, meeting changing needs $\rightarrow$ High-level service in new investor pools $\rightarrow$ Long-term efficiency and strong balance sheet $\rightarrow$ Develop, retain talent}}\\
	\hline
	\hline
	\multicolumn{2}{c}{\textbf{Clients}}\\
        \hline
	\hline
	\textbf{Client Type} & \textbf{2016 Share of AUM}\\
	\hline
	Insurance & 44\%\\
	Open-Ended Funds & 25\%\\
	Pension Funds & 15\%\\
	Other Institutional & 7\%\\
	Closed-Ended Funds & 6\%\\
	CB's \& Gov. Agencies & 3\%\\
	\hline
	$\Sigma$ (bn\pounds) & 283.7\\
	\hline
        \hline                                
        \multicolumn{2}{c}{\textbf{Investment Strategy}}\\
        \hline
        \hline 
	\multicolumn{2}{p{12cm}}{\textit{First-Hand Research $\rightarrow$ Team-Based Investing $\rightarrow$ Risk Focus $\rightarrow$ Long-Term View}}\\
	\hline
	\textbf{Investment Type} & \textbf{2016 AUM (bn\pounds)}\\
	\hline
	Equities & 89.1\\
	\textit{\% of Total} & 28.6\\
	Fixed Income & 70\\
	\textit{\% of Total} & 22.4\\
	Aberdeen Solutions & 134.5\\
	\textit{\% of Total} & 43.1\\
	Property & 18.5\\
	\textit{\% of Total} & 5.9\\
	\hline
	$\Sigma$ & 312.1\\
	\hline
	\hline        
        \hline                                
        \multicolumn{2}{c}{\textbf{Additional Financials}}\\ 
        \hline
        \hline
	\textbf{Statistic} & \textbf{2016 Value}\\
	\hline
	AUM (2017) & 308.1bn\pounds\\
	Outflows & 13.4bn\pounds\\
	Profit Before Tax & 115m\pounds\\
	Net Revenue & 483.6m\pounds\\
	Operating Expenses & 327.7m\pounds\\
	Cash @ Bank \& In-Hand & 838.1m\pounds\\
	(Cash \& Cash-Equivalents) & 548.8m\pounds\\
	Operating CF Before I\&T & 362.4m\pounds\\
	Dividend Payments & 280.4m\pounds\\
	\hline
        \multicolumn{2}{l}{\textit{Source: Prospectus \& Scheme Document}}\\
        \hline
	\caption{Aberdeen Asset Management PLC Information}
\end{longtable}
\end{center}

	\subsection*{Appendix 2: Fundamental Material}

\begin{table}[h]
        \scriptsize
	\renewcommand{\arraystretch}{1.25}
\begin{center}
\begin{tabular}{p{4.5cm}p{8cm}}
	\hline
	\multicolumn{2}{c}{\textbf{Basic Definitions}}\\
        \hline
	\textbf{Merger} & The approximately equivalent combination of two firms to form one larger firm.\\
	\textbf{Acquisition} & The takeover of one firm by another. Often \textit{harmonious} meaning both firms are in favour, for mutual benefit. Sometimes \textit{hostile}, where one firm wishes to add another to their portfolio often for single-firm gain.\\
	\hline
        \multicolumn{2}{c}{\textbf{Corporate Mechanics}}\\
        \hline
        \textbf{Horizontal} & Firms operating at an approximately equivalent level in their respective hierarchies, in the same industry.\\
        \textbf{Vertical} & Firms operating at different levels in their respective hierarchies/supply chains, in the same industry.\\
        \hline
        \multicolumn{2}{c}{\textbf{Industry Mechanics}}\\
        \hline
        \textbf{Conglomerate} & Firms in different industries with little/nothing in common; joining purely for their own expansion across markets.\\
        \textbf{Congereric `Product Extention'} & Firms that offer different products/services in the same market who wish to reduce the overall size of their industry thus, increasing thier own real estate of the market. Frequently becuase the two firms can benefit from similarities in their chain of operations.\\
        \textbf{Market Extension} & Firms that offer the same product/service in different markets, seeking larger overall real estate.\\
        \hline
	\multicolumn{2}{c}{\textbf{Economic Mechanics}}\\
	\hline
	\textbf{Economies of Scale} & Cost-Revenue optimization from efficiencies of size.\\
	\textbf{Economies of Scope} & Cost-Revenue optimization from efficiencies of diversification and expanded reach.\\
	\textbf{Industry Merger Wave} & Cyclical periods of higher merger volume within an industry, usually caused by macroeconomic states and factors (as explored in Table 9).\\
	\hline
	\multicolumn{2}{c}{\textbf{Synergies}}\\
	\hline
	\textbf{Merger Synergy} & One entity enhances another entity to create a total value which is greater than the sum of both individual entities.\\
	\textbf{Cost-Reduction} & Minimization of costs through efficiencies factors such as the removal of cross-overs and white spaces in operations.\\
	\textbf{Revenue-Enhancement} & Maximization of revenues through factors such as greater sales volume, market reach, and growth.\\
	\hline
\end{tabular}
        \caption{Merger Mechanics}
\end{center}
\end{table}

\newpage

\begin{table}[h]
        \scriptsize
	\renewcommand{\arraystretch}{1.25}
\begin{center}
\begin{tabular}{p{3cm}p{4.5cm}p{4.5cm}}
        & \textbf{Economies of Scale} & \textbf{Economies of Scope}\\
        \hline
	\textbf{Purpose} & Optimise the cost of production, focusing on one type of product. & Optimise the cost of production, focusing on multiple types of product.\\
	\textbf{Method} & Producing a bulk of one type of product. & Producing multiple types of products using cross-over efficiencies.\\
	\textbf{Strategy} & Standardization & Diversification\\
	\textbf{Product Example} & One firm's raw materials may aid the prodcution methods of another firm, vice-versa. Merge to decrease production time (efficiency). & One firm's resources may be able to be used to create different products of another firm, vice-versa. Merge to increase product variety (efficiency).\\
	\textbf{Service Example} & One firm may use distributors and transportation that is relevant to another firm and on which there is unoptimised space, vice-versa. Merge to decrease distribution time and costs (efficiency). & One firm may have connections in areas in which the other firm is relevant but not present, vice-versa. Merge to increase service variety (efficiency).\\
        \hline
\end{tabular}
        \caption{Economies of Scale \& Scope}
\end{center}
\end{table}

\newpage

\begin{center}
        \scriptsize
\begin{longtable}{p{1.5cm}p{7cm}p{4cm}}
        \textbf{Period} & \textbf{Description} & \textbf{Characteristics}\\
        \hline
        \multicolumn{3}{c}{\textit{Characteristics Include:}}\\
        \multicolumn{3}{c}{\textit{High Economic Growth Periods $\rightarrow$ Favourable Stock Prices $\rightarrow$ Technological Change}}\\ 
        \multicolumn{3}{c}{\textit{$\rightarrow$ Input Price Volatility $\rightarrow$ Legal \& Regulatory Changes $\rightarrow$ Financing Innovations}}\\
        \hline
        1895--1904 & The first major progression in infrastructure and production/manufacturing. There was a greater demand for the transportation of goods and people across borders so this called for increased rail infrastructure. Alongside this was the compliemntary advancements of electrical energy and its disribution of use across industries. The distribution of oil, metal ore, other ores, food  goods etc., was increasing immensely. So fourth, this period called for great economies of scale to deal with the demand and needs. This was a period in which brands, people and firms were defining themselves and their specialities so, desired large market reach. Hence, a monopolistic driver through horizontal mergers. & High economic growth period\newline Technological change\\
        1922--1929 & This was a period primarily made up of product extention. After various passings of new laws and regulation in reaction to the first wave, combatting monopolistic tendencies etc., this period saw many vertical mergers in an attempt to combine efficiencies of industries such as mining, ifrastructure, building etc. Thus, an oligopolistic period. There was a huge demand by firms for scope for mass distribution. & High economic growth period\newline Legal and regulatory changes\\
        1960's & This was a period of conglomorate mergers, known as the period where the market was `rewarding' diversification. There was a large opportunity at this point for large established firms to being acquiring other firms, not for vertical reasons discussed prior, but simply for more basic financial and economic reasons. Firms aimed to reduce instability by diversifying their scope of operations. This was especially apparent in low-growth-prospect markets. Companies aimed to satisfy wide post-World War II demands by broadening their offerings. This not only meant more revenue form more places but, a safety net in reaction to future changing demands as more businesses could support this. Convertable bonds also played a role. Frequently two firms with low growth prospects merged to imporve P/E and EPS, by focussing on decreasing the denominator of the latter. This was accomplished by issuing a substantial amount of convertable bonds which would not be classed as shares. This was adided by the driving of earnings by the merger also. However, various legal acts, including the Tax Reform Act of 1969, which put an end to manipulation of convertable bonds by requiring their accountance as if converted. Additionally, laws were passed regarding conglomerate logistics also. & Financing innovations\\
        1981--1989 & Known as the unwinding of the conglomerate wave and in response to the low economic grwoth during the 1970's. It was more of a reactive period than a proactive one; many of the 60's' congolerates had failed in the sense that the mergers had worrked in an inverse manner to that sought. In result, the sum of the individual parts of the merged firms were was greater than thier whole. Therefore, the reaction was to unwind this and operate firms in their stand-alone form. There was also large economic growth around this period. One part of this was the issuance of junk bonds, where firms with cash-raising problems issue bonds with low credit ratings and promise for high yield. These were designed to be bought in bulk as part of a `diverse' package. This attracted buyers and therefore raised cash, in a completely new market. It raised more capital than expected which went towards acquisitions. Many of these were `bustup' acquisitions; focusing the breaking up firms with part-sums greater than their whole. Pieces were sold off and revenue used to reduce debt. & High economic growth periods\newline Legal and regulatory change\newline Financing innovations\\
	1992--2000 & This period saw the innovation and rise of brand new firms and even industries/subindustries. For example, following the huge success with PS2 Terminals and Model M's in the 80's/90's, IBM began overtaking every workplace with the start of their famous T-Series of portable computers (ThinkPads), UltraDocks and business accessories for the pragmatic businessman. This was not only a new opportunity for many industries in manufacturing, engineering and computer science; it was also a huge one simply because of the percieved mass `need' for these business items, across most business platforms. Therefore, this was a period of P/E and EPS increae (attractiveness to investors and acquirers etc.) driven by the numerator of the latter; earnings. This expansion wasnt only happening in relative domestic markets, globalisation was rising; increasing requirement for scale in manufactuing and distribution etc. thus, mergers. A great example of network exapansion was the increase in demand for the AT\&T Merlin telecommunications system. This was a growth and earnings-driven period, not debt. & Favourable stock prices\newline Technological change\\
        \hline
        \caption{Merger Wave Timeline}
\end{longtable}
\end{center}

\newpage

\begin{center}
        \scriptsize
\begin{longtable}{p{5cm}p{7.5cm}}
        \multicolumn{2}{c}{\textit{Mergers are products of strategy and relationships, not quantitative methods}}\\
        \hline
        \hline
        \multicolumn{2}{c}{\textbf{Overview}}\\
        \hline
        \hline
	\multicolumn{2}{p{13cm}}{Theories of mergers to separate into three categories: [1] rationale behind the reasoning for a merger, [2] expected impact of the merger and, [3] the process/timeline over which the merger taes place (Weston, et al. 2011).\newline\newline As mergers are designed to add efficiency, wealth and reduce costs; there is a large focus on economies of scale and transaction costs (Leepsa, Mishra, 2016).\newline\newline Furthermore, Gohlich (2012) describes four primary theories of the merger process, rationale and impact: [1] synergy theory, [2] agency theory, [3] market power theory and, [4] strategic similarity theory.\newline\newline Additionally, Romano (1992) discusses the breakdown of the two primary explanations of mergers and acquisitions (value-maximising/non-value-maximising) into: [1] benefits of efficiency from synergies (tecnhnological harmony/development and economies of scale etc.); [2] financial benefits from tax, labour reconfiguration, etc.; [3] removal of market myopia (a na\"{i}ve approach in which firms over-concentrate efforts on singular or small operations), reducing productivity (which Bradley, et al. (1983) suggest that the latter can be heavily atrributed to inefficient managers and poor intra-firm and firm-to-market communications) and; (non-value-maximising factors) [1] diversification/intellectual growth, [2] self-promotion of power, [3] free cash flow, [4] `winner's curse' hypothesis (Varaiya, 1988) (in which firms overvalue aspects of a business or a company itself. For example, when acquiring, paying too-high-a-price thus, `winning' but similtaneously `losing').\newline\newline A basic view of a general merger rationale timescale is presented by Giannopoulos (2008), stating that pre-merger operations are focused on profit-increasing methods including market power, economies of scale, creating barrier for entry (especially in congeneric mergers where you operate for real estate). Post-merger operations re focsed on cost-reducing methods like asset re-alignment, resource management etc.\newline\newline Grouping much of the above are Motis' (2007) ideas of grouping pre-merger motives into `industrial organisation' and post-merger re-alignments into `corporate governance'. These essentially reflect the idea that pre-merger activities are based on power and profit aspirations and post-merger activities are focused on solving corporate problems between staff, their methods and the logistics of optimisation.\newline}\\
	\multicolumn{2}{p{13cm}}{Another relevant idea is one described by Gorton, et al. (2009) where merger motive can be hugeley driven by size. It's said that [1] smaller industries may experience a higher volume of mergers as firms see more potential for an easier route to higher industry real estate. [2] this idea of size can be what triggers industry waves as ther becomes competition for more tangible real estate. [3] larger companies over-pay in these situations as they approach from a more conglomerate point of view; `buying the market', instead of gaining relevant real estate such as the smaller firms at the same time.\newline\newline Overall, Coase (1937) gives the view that M\&A is value-increasing in the sense that efficency, technology enhancement and growth, scale and scope are greater than any problem in corporate governance etc. Jensen (1986) gives the opposing view that M\&A is value-reducing as managers frequently mis-value the porjected synergies of mergers and thus, mis-align cash post-merger. This is very much on a firm-by-firm, manager-by-manager basis however.}\\
        \hline
        \hline
        \multicolumn{2}{c}{\textbf{Efficiency Theory}}\\
        \hline
        \hline
        \multicolumn{2}{p{13cm}}{Wolfe, et al. (2011) highights the fundamentals of efficiency: optimising the use of and inter-linking nature of skills between the acquirer and target, repurposing and re-aligning resources in the supply chain and in the (post-merger) firm itself, sharing and building the technologies of each firm to perhaps make something `greater than he sum of its parts', eliminating cross-over expenses and promoting both firms' specialities; in effect increasing efficiency and reducing costs. It's suggested that firms with different strengths and weknesses offset those of each-other. Using this idea, many areas of operation may equalize/normalize; such as management, intra-firm operations and sourcing. These ideas therefore form much of the basis on which horizontal mergers are built.}\\
        \hline
        \multicolumn{2}{c}{\textbf{Efficiency Examples}}\\
        \hline
        \textbf{M\&A} & \textbf{Rationale}\\
        \hline
        Facebook \& Little Eye Labs & Significant enhancement of mobile development. Facebook provided the fundamental ground and content, Little Eye Labs provided the efficienct technology.\\
        Holcim \& ACC & An exchnage of mutually beneficial methodologies and technologies which is aimed at decreasing time-to-production and increase rate of innovation.\\ 
        United Breweries \& various other breweries & Minimising resource waste in a specific situation where the sum of waste of all the associated firms was enough to house the equivalent of another brewery. In effect, removing the equivalent cost of running one or more breweries post-merger.\\
        ABG Shipyard \& Western India Shipyard & Minimising repair, docking, maintenance costs etc.\\
	\hline
        \multicolumn{2}{p{13cm}}{\textit{Source: Leepsa, Chandra, (2016)}}\\
        \hline
        \hline
        \multicolumn{2}{c}{\textbf{Synergy Gain Theory}}\\
        \hline
        \hline
        \multicolumn{2}{p{13cm}}{The popular saying ``the whole is greater than the sum of its parts", refers to the fact that often the combination of two firms is greater than the sum of their efforts if they were operating side-by-side. This section relates heavily to economies of scale  through which fixed costs are distributed across a longer span. Further, economies of scope which allow resources of each firm to act as efficiency enhancers. For example, a firm who has produced an extremely effective rear-end SQL system (such as Amazon) would greatly benefit from a firm which specialises in front-end promotion.\newline\newline Economies of scale are driven by such an immense operation chain being simplified or optimisation of inventory holding. Economies of scope are driven primarily by automatic reductions in costs through widening of resources and expertise (Romano, 1992).}\\
        \hline
        \multicolumn{2}{c}{\textbf{Synergy Examples}}\\
        \hline
        \textbf{M\&A} & \textbf{Rationale}\\
        \hline
        Prism Cement \& Milano Bathroom Fittings & Increasing manufacturing capabilities, resources and real estate. Manufacturing plants also became more commonly and effectively accessible (for transport (import/exports etc.)).\\
        Steel Authority of India \& Neelachal Ispat Nigam & With gowth prospects in search of a wider product span and manufacturing capabilities of such. More effective applicaiton of manufacturing resources. Quality enhancements also as each firm added an aspect which benefitted a specific procedure.\\
	\hline
        \multicolumn{2}{p{13cm}}{\textit{Source: Leepsa, Chandra, (2016)}}\\
        \hline
        \hline
        \multicolumn{2}{c}{\textbf{Diversification Theory}}\\
        \hline
        \hline
        \multicolumn{2}{p{13cm}}{Most commonly firms aim to diversify in product range or geographical reach (Weston, et al., 2010). In many cases, this type of diversifiaction increases debt capacity and decreases tax liabilities. Also, spreading into a larger geographical area can also have great effects on reputation in different cultures etc. Often, these methods are considered to be better than intra-firm growth as there is much greater potential for exposure in the former.}\\
        \hline
        \multicolumn{2}{c}{\textbf{Diversification Examples}}\\
        \hline
        \textbf{M\&A} & \textbf{Rationale}\\ 
        \hline
        EID Parry \& Nutraceuticals Co. \& Valensa International & Greater access to the United States and European markets which act as primary hubs now for gloabl trade.\\
        Hindustan Unilever Ltd. \& International Bestfoods Ltd. & Entering smaller markets in search of reviving older and forgotten operations.\\
        Kamadgiri Fashion Ltd. \& Stripes Apparels Ltd. & Expanding capacity of production and reducing costs of such by using cheaper locations.\\
        \hline
        \multicolumn{2}{p{13cm}}{\textit{Source: Leepsa, Chandra, (2016)}}\\
        \hline
        \hline
        \multicolumn{2}{c}{\textbf{Strategic Realignment Theory}}\\
        \hline
        \hline
        \multicolumn{2}{p{13cm}}{Weston, et al. (2010) states highlights the fact that it is important for a firm to optimise their strategies and operation chains relative to the economic and technological state at the given time. Unlike long-run motives of M\&A, strategic drivers are response techniques which mutually benefit firms involved as they use their reseources to become more fficient in market reactions.}\\
	\hline
        \multicolumn{2}{c}{\textbf{Strategic Realignment Examples}}\\
        \hline
        \textbf{M\&A} & \textbf{Rationale}\\ 
        \hline
        Tata Motors Ltd. \& Tata Finance Ltd. & Gowing various support aspects of thier business in order to be more secure in a global market and better equipt for different market conditions.\\
        Merger of Gabriel India Ltd. \& Stallion Shox Ltd. & Modernise technologies and develop R\&D in order to better prepare for reactions to other firms and the differing states of economies in which they operate (e.g. different technological stages).\\
        Merger of Novartis India Ltd. \& Ciba CkdBiochem Ltd. & Stabilise quality and sustainability of manufacturing and research techniques and, sources.\\
        \hline
        \multicolumn{2}{p{13cm}}{\textit{Source: Leepsa, Chandra, (2016)}}\\
        \hline
        \hline
        \multicolumn{2}{c}{\textbf{Undervaluation Theory}}\\
        \hline
        \hline
        \multicolumn{2}{p{13cm}}{Firms may be targeted because of their undervaluation. This can be dominant in conglomerate acquisitions, for example. Weston, et al. (2010) highlights that historically, undervaluation is the doing of inefficient managers who don't realise a firm's potential in time. In a case where the acquirer has insider information, they have the best chances of realising and pockets of inefficiency. Most commonly, undervaluation is seen in the difference between market value and replacement costs of assets; where the cost of replicating it would be greater than the current valuation.}\\
        \hline
        \multicolumn{2}{c}{\textbf{Undervaluation Examples}}\\
        \hline
        \textbf{M\&A} & \textbf{Rationale}\\ 
        \hline
        \hline
        \multicolumn{2}{p{13cm}}{\textit{Source: Leepsa, Chandra, (2016)}}\\
        \hline
        \hline
        \multicolumn{2}{c}{\textbf{Market Power Theory}}\\        
        \hline
        \hline
	\multicolumn{2}{p{13cm}}{Many firms wish to dominate their industry/market. Weston, et al. (2010) says that increased market share is not always the best option for the market as a whole. This strategy leads to a high concentration of firms in the industry and lowering competition, particularly in waves. Recognised, is either a monopolistic nature or bloated competition between very large firms. These are both poor for the market, industry and economy in the long-run. Quality of products and services would become irrelevant as a result. Additionally, throughout large horizontal mergers (taking over an industry as described), as there is a decrease in the overall number of firms, a firm's reliance on itself will become far more important and its volatility will increase.}\\
        \hline
        \multicolumn{2}{c}{\textbf{Market Power Examples}}\\
	\hline
        \textbf{M\&A} & \textbf{Rationale}\\ 
        \hline
        Himadri Chemicals \& Industries Ltd. & Expanding into larger Asian markets, relative to areas of high demand and ease of production; makin the firm, as it has access to more capital than many other of its type with the same goals and geography, a market leader.\\
        Merger of Ultratech Cement Ltd. \& Samruddhi Cement Ltd. & Simply two of the largest firms in their industry in India, joining for an even bigger share and control of the market.\\
        \hline
        \multicolumn{2}{p{13cm}}{\textit{Source: Leepsa, Chandra, (2016)}}\\
        \hline
        \hline
        \multicolumn{2}{c}{\textbf{Tax \& Redistribution}}\\        
        \hline
        \hline
        \multicolumn{2}{p{13cm}}{Finally, Weston, et al. (2010) highlights firms' large desire to minimise tax liabilities. Many firms may not be looking for efficiency etc.; they may simply view mergers as ways of spreading tax. Ofr example, an acquirer may purchase a small growth firm with little liabilities etc., aiding capital gain tax substitutions. Or, a high-profit acquirer may purchase a low-profit firm in search of tax reduction again. Or, by acquiring firms with specific depreciable asset configurations.\newline\newline From a redistribution perspective, Ahern and Weston (2007) state that acquirers aim to use thier new acquisitions to reorganise their tax, bondholder, labour and pension cost configurations. This is commonly reflected in shareholder wealth redistribution. Tax and pension (etc.) redistribution comes from the government after previousl discussed stategies are in-play. There may also be a redistribution of employee costs to shareholder wealth, depending on the capital structure and payou policy of either firm etc. In this case, conclusions may be made that drivers for mergers come from s shareholder-driven point of view as opposed to economic efficiency.}\\
        \hline
        \multicolumn{2}{c}{\textbf{Tax \& Redistribution Examples}}\\
        \hline
        \textbf{M\&A} & \textbf{Rationale}\\ 
        \hline
        Merger of Indo Rama Synthetics \& Indo Rama Petrochemicals Ltd. & Bypass various taxes in the purchase of materials.\\
        Merger of Mirc Electronics Ltd. \& Onida Savak Ltd. & Income tax loss benefits and sales tax benefits.\\
        Neelachal Ispat Nigam Ltd. \& Konark Met Coke Ltd. & Saving sales tax on inter-firm sales between the two merged firms.\\
        Air India \& Indian Airlines & Reducing human labour per aircraft, reducing total costs and redistributing elsewhere.\\
        \hline
        \multicolumn{2}{p{13cm}}{\textit{Source: Leepsa, Chandra, (2016)}}\\
        \hline
        \caption{Theories of Mergers \& Acquisitions}
\end{longtable}
\end{center}

\newpage

\renewcommand\refname{Bibliography}

\begin{thebibliography}{9}

	\bibitem{a}
		Aberdeen Asset Management PLC. (2016).
		\textsl{Annual Report and Accounts 2016.}
		Available At: https://www.annualreports.com/HostedData/AnnualReports/PDF/LSE\_ADN\_ 2016.pdf. (Accessed: 26/01/2021).

	\bibitem{b}
		Aberdeen Standard Investments. (2017).
		\textsl{Merger of Aberdeen Asset Management PLC and Standard Life plc.}
		Available At: https://www.aberdeenstandard.com/ceo-letter. (Accessed: 26/01/2021).

	\bibitem{c}
		ADV Ratings. (2021).
		\textsl{Top UK Asset Managment Firms by AUM.}
		Available At: https://www.advratings.com/top-uk-asset-managers. (Accessed 27/01/2021).
	
	\bibitem{d}
		Ahern, K.R., Weston, J.F. (2007).
		\textsl{M\&As: the good, the bad, and the ugly.}
		Journal of Applied Finance. Volume 17, Number 1, Pages 5--20.

	\bibitem{e}
		Bradley, M., Desai, A. and Kim, E.H. (1983).
		\textsl{The rationale behind interfirm tender offers.}
		Journal of Financial Economics. Volume 11, Pages 183--206.

	\bibitem{f}
		Cheng, Y., Wickramanayake, J. and Sagaram, J.P.A (2007).
		\textsl{Acquiring Firms’ Shareholder Wealth Effects of Selected Asian Domestic and Cross-Border Takeover Bids: China and India 1999-2003.}

	\bibitem{g}
		CNBC. (2017).
		\textsl{Could 2018 surprise with the same outsized gains as 2017?.}
		Available At: https://www.cnbc.com/2017/12/26/could-2018-surprise-with-the-same-outsize-gains-as-2017.html. (Accessed: 10/02/2021).

	\bibitem{h}
		\textit{The} Economist. (2017).
		\textsl{Asset managers: The tide turns.}
		Available At: http://www.economist.com/news/finance-and-economics/21695552-consumers-are-finally-revolting. (Accessed: 11/02/2021).

	\bibitem{i}
        	Giannopoulous, M. (2013).
		\textsl{Divesting assets and redeploying resources as predictors of the performance of acquisitions: the case of Greece (Doctoral dissertation).}
		Available At: http://bura.brunel.ac.uk/bitstream/2438/8754/1/FulltextThesis.pdf. (Accessed 29/01/2021).

        \bibitem{j}
        	Gohlich, T.R. (2012).
		\textsl{The performance effects of mergers within the German Cooperative Banking Sector.}
		Available At: http://essay.utwente.nl/61530/. (Accessed 29/01/2021).

        \bibitem{k}
        	Gorton, G., Kahl, M. and Rosen, R.J. (2009).
		\textsl{Eat or be eaten: a theory of mergers and firm size.}
		Available At: http://www.afajof.org/afa/forthcoming/3015.pdf. (Accessed 29/01/2021).

        \bibitem{m}
		Leepsa, N.M., Mishra, C.S. (2016).
		\textsl{Theory and Practice of Mergers and Acquisitions: Empirical Evidence from Indian Cases.}
		IIMS Journal of Management Science. Volume 7, Number 2, Pages 179--194.

        \bibitem{n}
		Motis, J. (2007).
		\textsl{Mergers and Acquisitions Motives.}
		Available At: https://economics.soc.uoc.gr /wpa/docs/paper2mottis.pdf. (Accessed 29/01/2021).

        \bibitem{o}
        	NASDAQ. (2017).
		\textsl{15 of the Best Mergers \& Acquisitions of 2017.}
		Available At: https://www.nasdaq.com/articles/15-best-mergers-acquisitions-2017-2017-12-29. (Accessed: 10/02/2021).

        \bibitem{p}
		Romano, R. (1992).
		\textsl{A guide to takeovers theory, evidence, regulations.}
		The Yale Journal on Regulation. Volume 9, Pages 119--180.

	\bibitem{q}
        	Standard Life PLC. (2016).
		\textsl{Half year results 2016.}
		Available At: https://www.aberdeenstandard.com/docs?documentId=GB-280720-122264-1\&\_ga=2.9644469.1281101829.1611655633-798324784.1611401377. (Accessed: 26/01/2021).

        \bibitem{r}
        	Standard Life PLC. (2017).
		\textsl{Standard Life PLC (To Be Renamed Standard Life Aberdeen PLC).}

        \bibitem{s}
        	Statista. (2021). [1].
		\textsl{Merger and acquisitions (M\&A) transaction volume in the financial services sector in the United Kingdom (UK) from 2016 to 2019.}
		Available At: https://www.statista.com/statistics/411636/uk-m-and-a-deal-volume-financial-sector/. (Accessed: 10/02/2021).

        \bibitem{t}
        	Statista. (2021). [2].
		\textsl{Value of mergers and acquisitions (M\&A) worldwide from 1985 to 2020.}
		Available At: https://www.statista.com/statistics/267369/volume-of-mergers-and-acquisitions-worldwide/. (Accessed: 27/01/2021).

        \bibitem{u}
        	Varaiya, N. (1988).
		\textsl{The 'Winner's Curse' Hypothesis and Corporate Takeovers.}
		Managerial and Decision Economics. Volume 9, Number 3, Pages 209--219.

        \bibitem{v}
        	Weston, J.F., Mitchel, M.L., Mulherin, J.H. and Salwan, P. (2010).
		\textsl{Takeovers, restructuring, and corporate governance.}
		Fourth Edition, Pearson Education.

        \bibitem{w}
        	Weston et al. (2011).
		\textsl{Mergers, Restructuring, and Corporate Control.}
		Prentice Hall.

        \bibitem{x}
		Wolfe, M., Stressman, S. and Manfredo, M. (2011).
		\textsl{The acquisition of IBP by Tyson foods in 2001: pre and post merger financial performance.}
		American Journal of Agricultural Economics. Volume 93, Number 2, Pages 1--6.

\end{thebibliography}

\end{document}
