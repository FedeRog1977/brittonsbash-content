\documentclass[11pt, english]{article}              
        \usepackage{geometry}
                \geometry{                          
                        a4paper,total={210mm,297mm},
                        tmargin=40.8mm,
                        bmargin=40.8mm,
                        lmargin=32.6mm,
                        rmargin=32.6mm,
                }

        \usepackage{titlesec}         
                \titleformat{\section}
                        {\normalfont\fontsize{18}{16}\bfseries}{\thesection}{0.5em}{}
                \titleformat{\subsection}
                        {\normalfont\fontsize{14}{16}\bfseries}{\thesubsection}{1em}{}
                \titleformat{\subsubsection}
                        {\normalfont\fontsize{11}{16}\bfseries}{\thesubsubsection}{1em}{}

        \usepackage{longtable}
        \usepackage{multirow}

        \usepackage[labelfont=bf,textfont=bf,font=small,skip=8pt]{caption}

        \setlength{\parindent}{0pt}
        \renewcommand{\baselinestretch}{1.25}
       \usepackage{setspace}

        \usepackage{amsmath}
        \usepackage{amssymb}
        
        \usepackage{tikz}
                \usetikzlibrary{trees,arrows,topaths}

        \usepackage[utf8]{inputenc}
        \usepackage[official]{eurosym}

        \usepackage{graphicx}

\begin{document}

\pagenumbering{gobble}

        \title{\textsc{AG430 Corporate Financing\\ Coursework Examination}}
        \author{\textsc{Lewis Britton}}
        \date{\textsc{Academic Year 2020/2021}}
        \maketitle

\newpage

\pagenumbering{roman}

        \renewcommand{\contentsname}{Table of Contents} 

	\tableofcontents

\newpage

\pagenumbering{arabic}

\section{Payout Policy (Section 1)}

	\subsection{Question 1: Payout Policy \& Signalling}

	\textit{Fama and French (2001) report that the number and proportion of publicly listed US industrial firms paying a dividend declined significantly between 1978 and 1999. Critically evaluate the various explanations for this phenomenon using appropriate studies of corporate payout policy. Your evaluation should consider the evidence from financial markets in the United States, as well as international markets.}\\

	The `disappearing dividend phenomenon', also simply decreasing dividends, occurs when a firm ether stops paying dividends or loses the ability to do so. Losing the ability may stem from delisting or mergers etc.\\

	The primary study surrounding this issue is Fama and French’s (2001) investigation of the characteristics of firms who pay and firms who do not pay dividends. They state that, in the case of this study, they consider `non-payers' as firms who simply do not pay dividends and `payers' as: [1] new firms making an entry to dividend payments in their year of listing, [2] firms which are choosing to begin paying a dividend and, [3] firms resuming dividend payments. On the surface, Fama and French (2001) find that there is a decrease of 45.7\% in firms paying dividends over the period of 1978-1999. This study takes place over the NYSE, AMEX and NASDAQ. They outline findings that a firm with high profitability, great size, and weak investment opportunities are likely to be dividend payers. Inferred from this is low profitability, little size, and strong investment opportunity firms are likely non-payers. Table 1, below, outlines this in a more efficient manner. It is therefore assumed that these firms are the most evident in the `disappearing dividend phenomenon'.\\

	\begin{table}[h]
		\scriptsize
	\begin{center}
	\begin{tabular}{p{3.5cm}p{2.5cm}p{2.5cm}p{2.5cm}}
		\textbf{Firm Cat.} & \textbf{Profitability} & \textbf{Growth} & \textbf{Investment Opp.}\\
		\hline
		Dividend Payers & $\uparrow$ & $\uparrow$ & $\downarrow$\\
		Dividend Non-Payers & $\downarrow$ & $\downarrow$ & $\uparrow$\\
		\hline
	\end{tabular}
		\caption{Payer \& Non-Payer Characteristics}
	\end{center}
	\end{table}

		\newpage

		\subsubsection*{Evidential Findings}













\end{document}
