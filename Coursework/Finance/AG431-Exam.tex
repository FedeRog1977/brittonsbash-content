\documentclass[11pt, english]{article}
        \usepackage{geometry}
                \geometry{
                        a4paper,total={210mm,297mm},
                        tmargin=40.8mm,
                        bmargin=40.8mm,
                        lmargin=32.6mm,
                        rmargin=32.6mm,
                }

        \usepackage{titlesec}
                \titleformat{\section}
                        {\normalfont\fontsize{18}{16}\bfseries}{\thesection}{0.5em}{}
                \titleformat{\subsection}
                        {\normalfont\fontsize{14}{16}\bfseries}{\thesubsection}{1em}{}
                \titleformat{\subsubsection}
                        {\normalfont\fontsize{11}{16}\bfseries}{\thesubsubsection}{1em}{}

        \usepackage{longtable}
        \usepackage{multirow}

        \usepackage[labelfont=bf,textfont=bf,font=small,skip=8pt]{caption}
        
        \usepackage{hyperref}
                \hypersetup{
                        colorlinks=true,
                        linkcolor=black,
                        filecolor=magenta,
                        urlcolor=cyan,
                        }

        \setlength{\parindent}{0pt}
        \renewcommand{\baselinestretch}{1.25}
        \usepackage{setspace}

        \usepackage{amsmath}
        \usepackage{amssymb}

\begin{document}

\pagenumbering{gobble}

        \title{\textsc{AG431 Corporate Investment\\ Coursework Examination}}
        \author{\textsc{Lewis Britton}}
        \date{\textsc{Academic Year 2020/2021}}
        \maketitle

\newpage

\pagenumbering{roman}

        \renewcommand{\contentsname}{Table of Contents}

        \tableofcontents

\newpage

\pagenumbering{arabic}

\section{April 2018 Paper}

	\subsection{Question 1: Merger Theory}

	\textit{Discuss the main change forces driving mergers. What are the three theories of mergers and what are their implications for returns to bidders and target as well as combined returns.}

		\subsubsection*{Background}	

	\href{https://www.youtube.com/watch?v=0RMKEZzbkuo}{Mergers and acquisitions} don't only make excellent corporate titles on \href{https://www.youtube.com/watch?v=cISYzA36-ZY}{business cards}, they generally occur under two rational and credible primary motives. The first is when firms identify areas of operation in other firms which present opportunities for synergies; through which the \textit{whole} merged firm should be significantly greater than the \textit{sum-of-parts} of the stand-alone ``\href{https://www.youtube.com/watch?v=KBgrU4vX16I}{Walk-Alone}'' firms. Firms tend to ideantify relevant sets of rationale which allows them to develop strategies and propose synergies from the movement of a merger. Here, theories of mergers often predict their success or failure.\\

	There are various levels of mechanisms to a merger. It should be apparent, through \textit{corporate mechanics}, if the merger is horizontal or vertical. That is, whether it's between firms of an equivalent hierarchical level, in the same industry or; between firms at different levels, in the same industry. This may influence the \textit{industry mechanics} of the merger and how the acquisition proces will take place. It could be a conglomerate merger, where a firm generally acquires another to expand its own reach, commonly in a case where the firms have little or nothing in common. It may be congeneric, where firms offer different products/services and wish to increase their real estate in the market. This method often generates synergies from cross-over and efficiency enhancing factors. The merger may also be a market extension in firms which offer similar/the same products but in ifferent markets. These factors may also be influenced by motives behind the merger being hostile or friendly.\\

	Furthermore, the above is considered in the context of \textit{economic mechanics} including the cost-revenue optimization methods of focus on economies of scale and economies of scope. Factors and characteristics of merger waves may also influence the behaviour of firms in this context. That is, cyclical factors generated from, primarily, macroeconomic states can effect the decision making and required movements.

		\subsubsection*{Rationale: General Factors}

	One of the biggest drivers of mergers is the desire for talent acquisition. This could result in enhanced scale and scope of reputation and networks etc. In theory, this means further enhanced efficiency and investment. This is in hopes of increasing fixed/long-term realtionships and income. Talented and skilled people attract good business. This also holds promise for new perspectives and opinions on development in technological, logistical and risk-based areas.\\

	Linking to the above is specific specializations. Especially where you have one firm which is a leader in one sense and another firm which leads in another. Therefore, not only does the merger aid in developing business operation scope on a whole, it also gives a new persepctive on the enhanacements of existing specializations such as this as new perspectives arrive.\\

	Firms may merge to meet demand requirements. Especially in the case of large firms, joining forces only enhances this. The combination of talents should theoretically exponentially grow the merged firm meaning they can acquire more business outwith their current ones. It’s believed that the new firms are able to adapt to changing global economies more efficiently and on a larger scale, therefore having the knowledge, skills and ability to take on new types of business. This larger scale also should be able to produce better innovation and development.\\

	Firms also seeks widened brands and networks. When two reputable set of brands join, they tend to make equally, if not more, reputable and stable brands in the future. Therefore, leading to further improved networs, reputation and client relationships. Geographical expansion significantly aids this. This of course, is in search of a further diversified client network and scope.\\

	Additionally, firms seek to provide imporved choice which may prolong corporate life. Respective portfolios become more diversified and skills of employees become complimentary. Hence, a wider offering to combined clients. There's also a opportunity to eliminate overlap within specific operations here, meaning more effective concentration.

		\subsubsection*{Rationale: Autogenous Factors}

	Lipton (2006) describes the autogenous factors (factors which are determined within a firm) driving mergers.\\

	Firms wish to obtain as much market power as possible. It is said that this began during the 19$\mathrm{th}$ centruy oil and railroad mergers when firms required a great deal of market power across their industry in order to obtain, construct and porvide the resources necessary to operate as desired. These days, if government legislation and regulation were neglected, firms would often seek monopolistic power.\\

	Firms may merge to share benefits of imporved operating margin through reduced operating costs. Often a firm with a `good' operating margin will acquire a firm which has a lower operating margin; in search of using their resources to enhance operations. This synergy gap is said to sometimes generate value which covers the acquisition premium and deliver additional revenue, also. frequently, this process is aided when the acquirer repurposes assets and utilities of the adjacent firm to more efficient use. Additionally, the acquirer may also redistribute indirect costs, where much may have been put to poor or inefficicent use. This same principal goes for the removal of inefficient or redundant workforce.\\

	Firms may embrace the sharing of costs and benefits of the removal of excess. Upon merging, firms remove any corss-over or excess operational components from their chains, including supply and distribution etc. Of course, this has potential to significantly reduce fixed and recurring costs.\\

	Vertical integration through industry is often favoured. It is sometimes the case in a vertical merger that there are benefits to firms from different levels in the supply chain combining forces. This may be these case where, for example, a marine manufacturer may require production of specific components with huge premiums; it's cheaper for the . The same goes for some hardware and software producers.\\

	Firms may seek advantages of providing a more complete product line in order to be more competitive. This is often the case in firms who supply to large retailers and wish to keep their own supply as efficient as possible to effectively control purchasing costs and inventory management expenditure, while growing to provide with great scope.\\

	Firms wish to spread the risk of development and technological enhancement costs. The overuse of technology in many industries leaves many firms without the option of traditional business or e-business. This obviously requires huge investement, sometimes unecessary investment in the sense that many firms do not need the technological support which is desired by normies in society. But unfortunately, this `need' of people makes in necessary. Especially in the case of this type of firm, it can be risky venturing into technological integration or development as knowledge and expertise may be lacking. Merging in this case may acquire the necessary enhancements without the expenditure of traditional business methods, of the original firm. It will also reduce the risk of failure. From a slightly different point of view, firms may view some aspects of functional technology as beneficial. This may extend to telecommunications, network providers and hardware etc. This is different; the formerly discussed has a value purely based on the perception and needs held by society. The latter shows potential to enhance efficiency and reduce operating costs of business. Therefore, firms with less experience or otherwise-focussed operations may find it beneficial to merge with such technologically experienced firms.\\

	Firms will always attempt to respond to the global market changes. The desires of consumers aren't just changing in a firm's domestic surrounding, they are also changing globally, and in different ways. Firms which wish to `modernize' and integrate themselves into modern culture may wish to merge globally or with other firms who better-understand the relative societal surroundings, in order to deliver more relevant goods/services in the most efficient of fashions. Thus, saving potentially wasted investment in misunderstandings or not-fully-aligned product delivery. This motive is necessary if a firm wishes to remain or become competitive on a large scale.\\

	Firms also respond to deregulation for similar reasons. Deregulation has been seen to allow diversification through congeneric mergers. For example, during the deregulation period of the 1990's where there was a great volume of acquisition of investment banks and insurance companies by commercial banks, when operational restrictions on commercial banks were loosened. Primarily, especially in the case of banking, this widens scope to reduce risky ventures and may also provide income from more stable areas.\\

	There may be a change of corporate focus. Firms have often realised, for example in the 1990's, that it is less effectve to attmpt to operate and manage certain componenets of their business. Therefore, firms would start to break down and perhaps sell-off firms, brands, componenets etc., creating further grwoth opportunities through the merging of these removed components with firms which are better-equipped or more interested in enhancing them.\\

	Firms respond to industry conolidation. Often when an industry is enhancing scope and some firms are slow to follow, they must make the decision between becoming obsolete or choosing a network (being a `consilidator'). Frequently out of prolonging measures and safety, firms merge on a large scale during these scenarios.\\

	Firms may give in to the pressure of shareholders to increase the value of the shareholding. This could tie in to many of the formerly dicussed motives however now, the drive may be enhanced by additional and often unecessary pressure. Often firms have conducted sales and divestitures of non-essential business in addition to other more relelvant acquisitions. This of course, contributes to mergers from two points of view.

		\subsubsection*{Rationale: Exogenous Factors}

	Lipton (2006) also describes various exogenous factors (factors which are determined by industries, markets, economic states etc., on a larger scale) driving mergers.

	Pooling accounting is one of the biggest directly financial motives in mergers. In the 1990's it helped avoid traditionally necessary `goodwill amortization' thus, avoiding particular dilution of earnings. This was especially apparent when an acquired firm was accounted for as a purchased entity. From 2001, `purchase accounting' replaced this idea of pooling. which further amends and removes burdens associated with share repurchases and asset dispositions, post-merger.\\

	Furthermore, activist hedge funds and activist institutional investors can significantly influence how a firm operates their sales and acquisitions of business. Simply, they have the ability to exert much force on the board of directors if they, sometimes na\"{i}vely, belive otherwise to directors regarding the use, efficiency or relevance of components of operations.\\

	Traditionally, various governmental policies have held the ability to overpower, prohibit or delay mergers. In the mighty 1980, The Shermen Act was passed in Congress to ``preserve free trade and competition''. In addition to the Federal Trade Commission Act and the Clayton Act, the set of US Antitrust laws is formed. These allow and allocate the correct licences and freedoms and restrictions in order to allow firms to act correctly in merger activity.\\

	Frequently arbitrageurs, hedge fund activists, and activist institutional investors combine forces in order to encourage firms to seek mergers. The often provide liquidity required for shares of firms associated with mergers; buying up large amounts of stock.\\

	Currency fluctuations very frequently influence cross-border mergers. Firms which exist in countries with strong currencies tend to have greater effect in acquisitions than firms in countries with poor currencies. For example, the decline of the Euro in 2000 saw huge acquisition of European firms by US firms. And, the same again with a strong dollar and US firms' acquisition of Asian firms.\\

	Deregulation has aided the global movement towards capitalism and privitization of firms which over time, has led to an increased pool of eligible candidated for mergers. Otherwise, protectionism tends to prohibit mergers. Some, perhaps more in-touch with traditional values, of us view this as a positive as it is what protects the fundamental value and function of what was once good.\\

	The increase of merger `experts' in the field has also aided analysis, conception, valuation and execution of successful mergers due to greater focus. This often provides firms with the confidence they need to interface with a merger and its componenets; encouraging greater merger volume. This is generally done by bodies associated with investment banks. This is also aided by specialist lawyers and consultants etc., in the field.\\

	With reference to the first and second `merger waves', brand-new firms types were appearning on the market. New technology firms and firms alike were often seeking temporary growth then to be acquired. The present days sees that if a new-looking and innovative start-up is formed, frequently larger firms look to acquire these to `help them realise potential'. Well, to claim the benefits from their growth.

		\subsubsection*{Merger Theories}

	Theories of mergers to separate into three categories: [1] rationale behind the reasoning for a merger, [2] expected impact of the merger and, [3] the process/timeline over which the merger taes place (Weston, et al. 2011).\\

	As mergers are designed to add efficiency, wealth and reduce costs; there is a large focus on economies of scale and transaction costs (Leepsa, Mishra, 2016).\\

	Furthermore, Gohlich (2012) describes four primary theories of the merger process, rationale and impact: [1] synergy theory, [2] agency theory, [3] market power theory and, [4] strategic similarity theory.\\

	Additionally, Romano (1992) discusses the breakdown of the two primary explanations of mergers and acquisitions (value-maximising/non-value-maximising) into: [1] benefits of efficiency from synergies (tecnhnological harmony/development and economies of scale etc.); [2] financial benefits from tax, labour reconfiguration, etc.; [3] removal of market myopia (a na\"{i}ve approach in which firms over-concentrate efforts on singular or small operations), reducing productivity (which Bradley, et al. (1983) suggest that the latter can be heavily atrributed to inefficient managers and poor intra-firm and firm-to-market communications) and; (non-value-maximising factors) [1] diversification/intellectual growth, [2] self-promotion of power, [3] free cash flow, [4] `winner's curse' hypothesis (Varaiya, 1988) (in which firms overvalue aspects of a business or a company itself. For example, when acquiring, paying too-high-a-price thus, `winning' but similtaneously `losing').\\
	
	A basic view of a general merger rationale timescale is presented by Giannopoulos (2008), stating that pre-merger operations are focused on profit-increasing methods including market power, economies of scale, creating barrier for entry (especially in congeneric mergers where you operate for real estate). Post-merger operations re focsed on cost-reducing methods like asset re-alignment, resource management etc.\\

	Grouping much of the above are Motis' (2007) ideas of grouping pre-merger motives into `industrial organisation' and post-merger re-alignments into `corporate governance'. These essentially reflect the idea that pre-merger activities are based on power and profit aspirations and post-merger activities are focused on solving corporate problems between staff, their methods and the logistics of optimisation.

		\subsubsection*{Merger Theories: Efficiency Theory}

	Wolfe, et al. (2011) highights the fundamentals of efficiency: optimising the use of and inter-linking nature of skills between the acquirer and target, repurposing and re-aligning resources in the supply chain and in the (post-merger) firm itself, sharing and building the technologies of each firm to perhaps make something `greater than he sum of its parts', eliminating cross-over expenses and promoting both firms' specialities; in effect increasing efficiency and reducing costs. It's suggested that firms with different strengths and weknesses offset those of each-other. Using this idea, many areas of operation may equalize/normalize; such as management, intra-firm operations and sourcing. These ideas therefore form much of the basis on which horizontal mergers are built.

		\subsubsection*{Merger Theories: Synergy Gain Theory}

	The popular saying ``the whole is greater than the sum of its parts", refers to the fact that often the combination of two firms is greater than the sum of their efforts if they were operating side-by-side. This section relates heavily to economies of scale  through which fixed costs are distributed across a longer span. Further, economies of scope which allow resources of each firm to act as efficiency enhancers. For example, a firm who has produced an extremely effective rear-end SQL system (such as Amazon) would greatly benefit from a firm which specialises in front-end promotion.\\

	Economies of scale are driven by such an immense operation chain being simplified or optimisation of inventory holding. Economies of scope are driven primarily by automatic reductions in costs through widening of resources and expertise (Romano, 1992).

		\subsubsection*{Merger Theories: Diversification Theory}

	Most commonly firms aim to diversify in product range or geographical reach (Weston, et al., 2010). In many cases, this type of diversifiaction increases debt capacity and decreases tax liabilities. Also, spreading into a larger geographical area can also have great effects on reputation in different cultures etc. Often, these methods are considered to be better than intra-firm growth as there is much greater potential for exposure in the former.

		\subsubsection*{Merger Theories: Strategic Realignment Theory}

	Weston, et al. (2010) states highlights the fact that it is important for a firm to optimise their strategies and operation chains relative to the economic and technological state at the given time. Unlike long-run motives of M\&A, strategic drivers are response techniques which mutually benefit firms involved as they use their reseources to become more fficient in market reactions.

		\subsubsection*{Merger Theories: Undervaluation Theory}

	Firms may be targeted because of their undervaluation. This can be dominant in conglomerate acquisitions, for example. Weston, et al. (2010) highlights that historically, undervaluation is the doing of inefficient managers who don't realise a firm's potential in time. In a case where the acquirer has insider information, they have the best chances of realising and pockets of inefficiency. Most commonly, undervaluation is seen in the difference between market value and replacement costs of assets; where the cost of replicating it would be greater than the current valuation.

		\subsubsection*{Merger Theories: Market Power Theory}

	Many firms wish to dominate their industry/market. Weston, et al. (2010) says that increased market share is not always the best option for the market as a whole. This strategy leads to a high concentration of firms in the industry and lowering competition, particularly in waves. Recognised, is either a monopolistic nature or bloated competition between very large firms. These are both poor for the market, industry and economy in the long-run. Quality of products and services would become irrelevant as a result. Additionally, throughout large horizontal mergers (taking over an industry as described), as there is a decrease in the overall number of firms, a firm's reliance on itself will become far more important and its volatility will increase.

		\subsubsection*{Merger Theories: Tax \& Redistribution}

	Finally, Weston, et al. (2010) highlights firms' large desire to minimise tax liabilities. Many firms may not be looking for efficiency etc.; they may simply view mergers as ways of spreading tax. Ofr example, an acquirer may purchase a small growth firm with little liabilities etc., aiding capital gain tax substitutions. Or, a high-profit acquirer may purchase a low-profit firm in search of tax reduction again. Or, by acquiring firms with specific depreciable asset configurations.\\

		From a redistribution perspective, Ahern and Weston (2007) state that acquirers aim to use thier new acquisitions to reorganise their tax, bondholder, labour and pension cost configurations. This is commonly reflected in shareholder wealth redistribution. Tax and pension (etc.) redistribution comes from the government after previousl discussed stategies are in-play. There may also be a redistribution of employee costs to shareholder wealth, depending on the capital structure and payou policy of either firm etc. In this case, conclusions may be made that drivers for mergers come from s shareholder-driven point of view as opposed to economic efficiency.

	\newpage 

	\subsection{Question 2: Payment \& Bidders}

	\textit{What are the empirircal findings relating to how the method of payment and the number of bidders affects bidder returns in a merger? With reference to the literature, discuss the empirical evidence on the combined merger returns predictions of the three main merger theories.}\\

	Piss off.

	\newpage

	\subsection{Question 3: Merger Waves}

	\textit{Compare and contrast the merger wave of the 1980’s with that of the 1990’s. With reference to the literature in this area, describe the endogenous factors that tend to influence mergers and could also lead to merger waves.}\\

	\textit{This question has been extended to compare and contrast all merger wave periods, as opposed to those listed in the above question.}

		\subsubsection*{Background}

	It is argued that there are five observed `merger waves', with an arguable sixth and seventh in more recent times. `Merger waves' are defined as a temporary period in which an industry or market etc. experiences substancially increased merger volume. Over these periods, merger waves have generally presented and/or been present within a set of six primary characteristics. These include: periods of high economic growth, contexts of favourable stock prices, periods of technological change, input price volatility, periods of legal and regulatory change, and innovations in financing and accounting methodologies.

		\subsubsection*{Wave of 1895--1904}

	This first wave of high merger volume was observed at the first major progression in infrastructure, production and manufacturing etc. There was a greater demand for the transportation of goods and people across borders so this called for increased rail infrastructure. Alongside this was the compliemntary advancements of electrical energy and its disribution of use across industries. The distribution of oil, metal ore, other ores, food goods etc., was increasing immensely. So fourth, this period called for great economies of scale to deal with the demand and needs. This was a period in which brands, people and firms were defining themselves and their specialities so, desired large market reach. Hence, a monopolistic driver through horizontal mergers. Evident from this, the relelvant industry/market characteristic shown in this period reflected a period of high economic growth and technological change. Lipton (2006) argues that `The Panics' of 1904 and 1907 and The Great War were the primary factors which had input in putting an end to this wave. In 1904, the Supreme Court made the decision to make amended antitrust laws applicable to horizontal mergers, in effect making mergers of this period and of these characteristics less desirable.

		\subsubsection*{Wave of 1922--1929}

	The merger wave of 1922--1929 was a a period primarily made up of product extention. After various passings of new laws and regulation in reaction to the first wave, combatting monopolistic tendencies etc., this period saw many vertical mergers in an attempt to combine efficiencies of industries such as mining, ifrastructure, building etc. Thus, an oligopolistic period. There was a huge demand by firms for scope for mass distribution. Lipton (2006) observes and argues that this is the period in which automobile and other mobile manufacturers made a name for themselves. For example, Ford became fully integrated with its supply and distribution; steel mills, railroads, factories, docks and shipyards etc. Thus overall, this period observed reaction to a period of economic growth and legal and regulatory changes.

		\subsubsection*{Wave of 1960's}

	The 1960's was a period dominated by conglomorate mergers. Lipton (2006) argues this period of high volume was apparent between 1955 and its floating end of 1969--1973, however it's not completely certain exactly how long this period lasted. It is known as the period where the market was ‘rewarding’ diversification. There was a large opportunity at this point for large established firms to being acquiring other firms, not for vertical reasons discussed prior, but simply for more basic financial and economic reasons. Firms aimed to reduce instability by diversifying their scope of operations. This was especially apparent in low-growth-prospect markets. Companies aimed to satisfy wide post-World War II demands by broadening their offerings. This not only meant more revenue form more places but, a safety net in reaction to future changing demands as more businesses could support this.\\

	Convertable bonds also played a role. Frequently two firms with low growth prospects merged to imporve P/E and EPS, by focussing on decreasing the denominator of the latter. This was accomplished by is- suing a substantial amount of convertable bonds which would not be classed as shares. This was adided by the driving of earnings by the merger also. However, various legal acts, including the Tax Reform Act of 1969, which put an end to manipulation of convertable bonds by re- quiring their accountance as if converted. Additionally, laws were passed regarding conglomerate logistics also. Evidently, the driving characteristic in this wave was innovation of financing methods. Lipton (2006) finds that a stock crash in these diversified conglomerates started in 1969 and lasted through to the early 70's, seeing the end of this wave. They also suggest that many of the firms did not fully realise the full potential of their diversification due to lacking liquidity and investment post-crash.

		\subsubsection*{Wave of 1981--1989}

	This period is known as the unwinding of the conglomerate and diversification wave. It is also in the context of response to low economic growth in the 1970's, meaning motives for this wave began to form in the mid-70's. However, Lipton (2006) argues that its volumetric scale took formed from 1984--1989. This is convenient as it echos the immense passion, drive, motivation and success of the iconic 1984--1989 television masterpiece; a Michael Mann classic, the stomping ground of Jan Hammer, the birthplace of Don Johnson and Philip Michael Thomas, the home of the long, cold, intimidating stares of Lieutenant Castillo; \textit{Miami Vice}.\\

	This wave was more of a reactive period than a proactive one; many of the 60's' congolerates had failed in the sense that the mergers had worrked in an inverse manner to that sought. In result, the sum of the individual parts of the merged firms were was greater than thier whole. Therefore, the reaction was to unwind this and operate firms in their stand-alone form. There was also large economic growth around this period. One part of this was the issuance of junk bonds, where firms with cash-raising problems issue bonds with low credit ratings and promise for high yield. These were designed to be bought in bulk as part of a `diverse' package. This attracted buyers and therefore raised cash, in a completely new market. It raised more capital than expected which went towards acquisitions. Many of these were `bustup' acquisitions; focusing the breaking up firms with part-sums greater than their whole. Pieces were sold off and revenue used to reduce debt.\\

	Much of the reaction in this period was also related to hostile takeovers. Lipton (2006) argues that this aspect reaches back to 1974 when the first `major firm' hostile bid was made by Morgan Stanley, on behalf of Inco. It's said that this then opened the floodgates to investment banks' making hostile bids and successful hostile takeovers. This hostile nature wasn't just the case in financial institutions, however. It was present in many manufacturing contexts for example, with the iconic \textit{Dornaus \& Dixon Bren Ten} 10mm police and government issue handgun/sidearm; where a financially and logistically suitable solution to the issue they faced with the subcintractor who supplied the gun's magazines was to forceably acquire they operations. This provided fast return to market. This was also the handgun used by Detective James ``Sonny'' Crockett so of course, they had to maintain their image and reuptation as a reliable firearm manufacturer.\\

	Primarily, this wave was driven by characteristics including high economic growth, legal and regulatory change and further financing innovations. Lipton (2006) argues that it began to come to a close in 1987 after the stock market crash (echoing the declined quiality in Dick Wolf's takeover from Michael Mann as Executive Producer of season 4 of \textit{Miami Vice} in 1897). It ended with the \$25 billion RJR Nabisco LBO and the fall of the junk bond market, in 1989 (echoed in the emotional decelleration towards the end of season 5 of \textit{Miami Vice} and the weak knees which came with the final hearing of \textit{Crockett's Theme} during that iconic hand shake in the 2-hour final\'{e}, \textit{Freefall}). This decline is said to be attributed also to the collapse of savings and loan banks and the capital problems faced by commercial banks.

		\subsubsection*{Wave of 1992--2000}

	The merger wave of the 1992--2000 period saw the innovation and rise of brand new firms and even industries/subindustries. For example, following the huge success with PS2 Terminals and Model M’s in the 80’s/90’s, IBM began taking over every workplace with the start of their famous T-Series and X-Series of portable computers (ThinkPads), UltraBases, UltraDocks and business accessories for the pragmatic businessman. This was not only a new opportunity for many industries in manufacturing, engineering and computer science; it was also a huge one simply because of the percieved mass `need' for these business items, across most business platforms. Therefore, this was a period of P/E and EPS increae (attrac- tiveness to investors and acquirers etc.) driven by the numerator of the latter; earnings. This expansion wasnt only happening in relative domestic markets, globalization was rising; increasing requirement for scale in man- ufactuing and distribution etc. thus, mergers. A great example of network exapansion was the increase in demand for the AT\&T Merlin telecommunications system. This was a growth and earnings-driven period, not debt.\\

	This period is referred to by Lipton (2006) as the `mega-deal era'. Firms held the belief in this context that `size matters'. This saw already-established, large firms become even more bullish. There was a preception that to continue to trade on an international scale, firms had to be of great size. In many cases, this led to the merging of firms which traditionally would not seem to fit together. Its oberved that these unalike mergers were not always carried out on purely fundamental grounds, for example, like in the rational reasoning of the joining of unalike firms in many vertical mergers. This may bot even have just been because firms were unalike however, there may simply just not have been a need for, or any substantial synergies in, the mergers. Lipton (2006) argues that the mergers of Citibank and Travelers, Chrysler and Daimler Benz, Exxon and Mobil, Boeing and McDonnell Douglas, AOL and Time Warner, and Vodafone and Mannesmann, etc., fit into this categorization.\\

	Merger volume rose from \$342 billion in 1992 to \$3.3 trillion in 2000, with 90\% of the largest mergers taking place from 1998--2000. The AOL and Time Warner merger holds the record at a value of \$165 billion. Mergers during this period cleanrly saw signs of motivation from technological change and favourable stock prices. Lipton (2006) claims its end came with the burst of the Millenium Bubble after scams and scandals such as those of Enron. In many cases, it's argued that lessons learned during the closing period of this wave provided much of the knowledge firms now apply to practice in corporate governance etc. The sad decelleration of telecommunications and technological business saw a large decline in tech merger volume and in effect, saw a fall of \~{}50\% in the NASDAQ from its 90's high. Subsequently, the junk bond market became close to non-existent, many lenders constricted requirements and standards, and mergers were generally not observed as very positive things for a while after this point.

		\subsubsection*{Rationale: Endogenous Factors}

        Lipton (2006) describes the endogenous factors (factors which are determined within a firm) driving mergers.\\

        Firms wish to obtain as much market power as possible. It is said that this began during the 19$\mathrm{th}$ centruy oil and railroad mergers when firms required a great deal of market power across their industry in order to obtain, construct and porvide the resources necessary to operate as desired. These days, if government legislation and regulation were neglected, firms would often seek monopolistic power.\\

        Firms may merge to share benefits of imporved operating margin through reduced operating costs. Often a firm with a `good' operating margin will acquire a firm which has a lower operating margin; in search of using their resources to enhance operations. This synergy gap is said to sometimes generate value which covers the acquisition premium and deliver additional revenue, also. frequently, this process is aided when the acquirer repurposes assets and utilities of the adjacent firm to more efficient use. Additionally, the acquirer may also redistribute indirect costs, where much may have been put to poor or inefficicent use. This same principal goes for the removal of inefficient or redundant workforce.\\

        Firms may embrace the sharing of costs and benefits of the removal of excess. Upon merging, firms remove any corss-over or excess operational components from their chains, including supply and distribution etc. Of course, this has potential to significantly reduce fixed and recurring costs.\\

        Vertical integration through industry is often favoured. It is sometimes the case in a vertical merger that there are benefits to firms from different levels in the supply chain combining forces. This may be these case where, for example, a marine manufacturer may require production of specific components with huge premiums; it's cheaper for the . The same goes for some hardware and software producers.\\

        Firms may seek advantages of providing a more complete product line in order to be more competitive. This is often the case in firms who supply to large retailers and wish to keep their own supply as efficient as possible to effectively control purchasing costs and inventory management expenditure, while growing to provide with great scope.\\

        Firms wish to spread the risk of development and technological enhancement costs. The overuse of technology in many industries leaves many firms without the option of traditional business or e-business. This obviously requires huge investement, sometimes unecessary investment in the sense that many firms do not need the technological support which is desired by normies in society. But unfortunately, this `need' of people makes in necessary. Especially in the case of this type of firm, it can be risky venturing into technological integration or development as knowledge and expertise may be lacking. Merging in this case may acquire the necessary enhancements without the expenditure of traditional business methods, of the original firm. It will also reduce the risk of failure. From a slightly different point of view, firms may view some aspects of functional technology as beneficial. This may extend to telecommunications, network providers and hardware etc. This is different; the formerly discussed has a value purely based on the perception and needs held by society. The latter shows potential to enhance efficiency and reduce operating costs of business. Therefore, firms with less experience or otherwise-focussed operations may find it beneficial to merge with such technologically experienced firms.\\

	Firms will always attempt to respond to the global market changes. The desires of consumers aren't just changing in a firm's domestic surrounding, they are also changing globally, and in different ways. Firms which wish to `modernize' and integrate themselves into modern culture may wish to merge globally or with other firms who better-understand the relative societal surroundings, in order to deliver more relevant goods/services in the most efficient of fashions. Thus, saving potentially wasted investment in misunderstandings or not-fully-aligned product delivery. This motive is necessary if a firm wishes to remain or become competitive on a large scale.\\

        Firms also respond to deregulation for similar reasons. Deregulation has been seen to allow diversification through congeneric mergers. For example, during the deregulation period of the 1990's where there was a great volume of acquisition of investment banks and insurance companies by commercial banks, when operational restrictions on commercial banks were loosened. Primarily, especially in the case of banking, this widens scope to reduce risky ventures and may also provide income from more stable areas.\\

        There may be a change of corporate focus. Firms have often realised, for example in the 1990's, that it is less effectve to attmpt to operate and manage certain componenets of their business. Therefore, firms would start to break down and perhaps sell-off firms, brands, componenets etc., creating further grwoth opportunities through the merging of these removed components with firms which are better-equipped or more interested in enhancing them.\\

        Firms respond to industry conolidation. Often when an industry is enhancing scope and some firms are slow to follow, they must make the decision between becoming obsolete or choosing a network (being a `consilidator'). Frequently out of prolonging measures and safety, firms merge on a large scale during these scenarios.\\

        Firms may give in to the pressure of shareholders to increase the value of the shareholding. This could tie in to many of the formerly dicussed motives however now, the drive may be enhanced by additional and often unecessary pressure. Often firms have conducted sales and divestitures of non-essential business in addition to other more relelvant acquisitions. This of course, contributes to mergers from two points of view.
	
	\newpage

	\subsection{Question 4: LBO Operation}

	\textit{Describe the various stages of a typical LBO operation and the sources of gains from an LBO. Why is the usual method of calculating the weighted average cost of capital unsuitable for a leveraged buyout? Outline the capital cash flow method for valuing a leveraged buyout and indicate how it is different from the usual DCF valuation for a firm.}\\

	Piss off.

	\newpage

	\subsection{Question 5: Merger Firm Value}

	\textit{
		\begin{itemize}
		\setlength\itemsep{0cm}
			\item[$a)$] Discuss the comparable transaction approach for valuing companies in the context of a merger. [30\%]
			\item[$b)$] Outline the steps involved in estimating the synergy of an acquisition. Assume that the firms have the same pre-tax cost of debt, the same debt-to-value ratio, the merged entity will have the same debt-to-value ratio and the tax rates for the merging and merged firms are all the same. Assume an initial growth phase of high growth driven by high reinvestment rates followed by a terminal phase of lower growth. [70\%]
		\end{itemize}
	}

		\subsubsection*{Background (a)}
        
        Valuing firms can be difficult in the context of a merger. Depending on the firm of type of person relevant, there may be different logical and philosophical views taken regarding how the value of a firm should be observed. Generally, we refer to two primary areas when discussing this; comparables, such as comparing transactions, and discounted cash flow, which accounts for many of the algebraic numberic methods used to generate estimations in capital budgeting, capital structure etc. Of course, these methods are usd to accurately forecast and target mergers and acquisitions, and aid in decision making to increase the subsequent value of a merged firm.

                \subsubsection*{Logic (a)}

        The comparable transaction approach is most commanly a traditional and logical approach to valuing and comparing firms which are being targeted for acquisition. If the targets of the acquisition has a similar business model to the acquirer, this portocol is relevant. This means that when evaluating a firm, their clients, products, size of transactions, operations, capital structure, trends and prospects, etc., must be somewhat relevant. According ratios are generated for the evaluated firms as a result. This approach is often referred to as the `common sense' approach as it uses accurate and fair market data and trends to generate decision making influence. This extends to the idea that amateur analysts and experienced often gather the relevant data from similar places. That is, public releases of the targets, databases such as Bloomberg's, CRPS etc., 8-K reports, and research and development notes of targets.\\

        If valuers seek the most accurate valuation of a firm, they must account for contexts in which there are [1] transactions from very similar, if not identical assets, of firms; [2] transactions closer to the data of valuation, simply to provide an accurate and up-to-date representation of the market; [3] transactions from assets which act in similar capacity and are approximately at the same level of `tangibility' across firms; [4] sufficient and similar types of information surrounding assets and transactions across firms, which originate from credible sources; and [5] ensuring transactions are real and not `intended'.

                \subsubsection*{Ratios \& Comparables (a)}
        
        We tend to observe ratios which are traditional indicators of success. These often inclue enterpise-value-to-sales, enterprise-value-to-EBITDA, price-to-earnings, paid-to-salespaid-to-book, paid-to-net-income, etc. Across relevant firms, these ratios are averaged to produce a \textit{transaction multiple} which are given as the coefficient to the according values from a single target firm; generating a `normalized' equity value. These are then averaged to produce an `accurate' comparable equity value. An example is displayed as follows:

	\begin{table}[h]
                \scriptsize
                \renewcommand{\arraystretch}{1.25}
        \begin{center}
        \begin{tabular}{p{4cm}|p{2cm}p{2cm}p{2cm}|p{2cm}}
                \multirow{2}{*}{\textbf{Ratio}} & \multicolumn{4}{c}{\textbf{Firm \& Comparative Value}}\\
                \cline{2-5}
                & \textbf{Amoco} & \textbf{Texaco} & \textbf{Conoco} & \textbf{Average}\\
                \hline
                Total Paid : Sales & 1.38 & 0.77 & 0.37 & 0.84\\
                Total Paid : Book & 3.00 & 2.79 & 2.29 & 2.69\\
                Total Paid : Net Income & 22.46 & 15.46 & 7.60 & 15.18\\
                Premium Paid, \% Target & 22.3\% & 17.7\% & 0.0\% & 13.3\%\\
                Premium Paid, \% Combined & 7.7\% & 6.3\% & 0.0\% & 4.7\%\\
                \hline
        \end{tabular}
                \caption{Sample Comparable Transactions}
        \end{center}
        \end{table}

        \begin{table}[h]
                \scriptsize
                \renewcommand{\arraystretch}{1.25}
        \begin{center}
        \begin{tabular}{p{3.5cm}|p{2cm}p{4.5cm}|p{2cm}}
                \multirow{2}{*}{\textbf{Component}} & \multicolumn{3}{c}{\textbf{Firm \& Comparative Value}}\\
                \cline{2-4}
                & \textbf{Mobil} & \textbf{Average Transaction Multiple} & \textbf{Equity Value}\\
                \hline
                Past 12-Month Sales & \$63.0 & 0.84 & \$53.0\\
                Book Value & \$19.0 & 2.69 & \$51.2\\
                Past 12-Month Net Income & \$2.9 & 15.18 & \$43.8\\
                Market Value Target & \$58.7 & 13.3\% & \$66.5\\
                Market Value Combined & \$233.7 & 4.7\% & \$69.6\\
                \cline{4-4}
                Average & \multicolumn{2}{c|}{} & \$56.8\\
                \hline
        \end{tabular}
                \caption{Sample Average Transaction Multiples}
        \end{center}
        \end{table}

                \subsubsection*{Positives \& Negatives}

        Conclusions from this method are based on freely available public information, meaning they should be consistently reliable and credible. It can provide a good reference of industry-level multiplpes at various period in market cycles. The method can also highlight the types of transactions which are relevant to current firms and buyers, and their asset requirements. However, this emthod can sometimes be na\"{i}ve in the sense that it may not account for varying conditions under which some transactions may have occured. It also doesn't focus on the value of potential synergies and other relevant intangibles. Thus, it's argued that these comparables frequently don't account for growth of revenue, risk, stage of a firm in its cycle, competition, and expansion etc.

		\subsubsection*{Background (b)}

        Valuing synergies is a crucial component of establishing the value of a merger as essentially, the merger's future value is based on how the adjoined firm will use synergies to enhance the growth rate of current operations and assets. Based on the types of synergies proposed in the merger, the value of the merged firm may change over time depending on: an daltered revenue growth rates from \textit{growth synergies}; altered margins due to synergies related to \textit{economies of scale and scope} etc.; altered tax rates due to tax beneits proposed in \textit{tax synergies}; altered costs of financing (debt and equity) due to \textit{financing synergies}; altered debt ratio due to change in risk attitude. Generating these factors on top of the proposed combined firm yields the total value of the combined firm with synergies. To observe the value of the synergies, the value of the (to-be) acquired firm and its control premium, and the value of the (to-be) acquirer are to be removed from the total value. Note that a control premium accounts for the value of a firm if it were to be optimized from the sense of investment, financing and dividend policies/decisions.

                \subsubsection*{Total Value of Combined Firm: Outline (b)}

        As noted, the first step is to value the entirity of the combined firm with the proposed synergies integrated. When doing this, we primarily look at \textit{operating synergies} and \textit{financial synergies}. These vary depending on the nature of the merger. For example, a horizontal merger is likely to see increased economies of scale, increased market power thus, reduced costs and inproved margins. Vertical mergers mostly see operational synergies form efficiencies in production chains and enhanced scope. So forth, firms must consider [1] where imporved margins come from (i.e. dominantly reduced costs or raised revenues); [2] if there is grwoth potential to the synergies or if they are more immediately functional; and [3] when the costs and the gains synergies will begin to take effect. This could also relate to the former two as expected expenditures and revenues may be effected by the ability produce synergy realization fast. Thus, from this and the formerly discussed, for ..., we know:

        $$V_S=V_{Combined\ w/\ S}-V_{Combined\ w/o\ S}$$
        $$\therefore V_S=V_{Combined\ w/\ S}-(V_{i_{Pre-Merger}}+V_{j_{Pre-Merger}})$$

                \subsubsection*{Total Value of Combined Firm: Steps (b)}

        Recall that in this scenario, firms should have equivalent pre-tax cost of debt, debt-to-value ratio, and this debt-to-value ratio holds for the merged firm. Tax rates remain the same. There's initial high growth stimulated by reinvestment followed by lower growth. Steps of valuation are outlined below.

	\begin{enumerate}
	\setlength\itemsep{0cm}
                \item Inputs
                \begin{enumerate}
                        \item Declare risk-free rate ($R_f$) and market risk premium ($R_m-R_f$)
                        \item Declare beta values ($\beta$) for each firm
                        \item Declare risk premium for firms ($\beta(R_m-R_f)$)
                        \item Declare pre-tax cost of debt ($r_D$), tax rate ($T$), debt-to-captial ratio ($w_D=\frac{Debt}{Debt+Equity}$), equity-to-capital ratio ($w_E=\frac{Equity}{Debt+Equity}$)
                        \item Declare revenues, EBIT; calculate collective revenue, EBIT
                        \item Declare pre-tax return in capital ($ROC_{\textrm{Pre-Tax}}$), reinvestment rate, length of initial high growth period ($N$ for time $t\in\{1,...,N\}$)
                \end{enumerate}
                \item Outputs
                \begin{enumerate}
                        \item Calculate cost of equity ($r_E=(R_f+\beta)(R_m-R_f)$)
                        \item Calculate post-tax cost of debt ($r_{D_{\textrm{Post-Tax}}}=r_D(1-T)$)
                        \item Calculate cost of capital
                        \begin{enumerate}
                                \item $WACC=r_Dw_D(1-T)+r_Ew_E$
                                \item $\therefore WACC=r_D\left(\frac{Debt}{Debt+Equity}\right)(1-T)+r_E\left(\frac{Equity}{Debt+Equity}\right)$
                        \end{enumerate}
                        \item Calculate post-tax return on capital
                        \begin{enumerate}
                                \item $ROC_{\textrm{Post-Tax}}=ROC_{\textrm{Pre-Tax}}(1-T)$
                        \end{enumerate}
                        \item Calculate expected growth rate for period $N$ of high growth
                        \begin{enumerate}
                                \item $g^e=ROC_{\textrm{Post-Tax}}(Reinvestment\ Rate)$
                        \end{enumerate}
                \end{enumerate}
                \item Value of Firms
                \begin{enumerate}
                        \item Calculate the present value of free cash flows (FCF) for period of high growth 
                        \begin{enumerate}
                                \item $PV_{FCF}=\sum_{N=1}^N(EBIT(1-T)(1-Reinvestment\ Rate))((1+g^e))\left(\frac{\frac{1-(1+g^e)^N}{(1-WACC)^N}}{WACC-g^e}\right)$
                        \end{enumerate}
                        \item Calculate terminal value
                        \begin{enumerate}
                                \item $TV=\sum_{N=1}^N(EBIT(1-T))((1+g^e)^N)(1+R_f)\left(\frac{\frac{1-R_f}{WACC}}{WACC-R_f}\right)$
                        \end{enumerate}
                        \item Calculate enterprise value
                        \begin{enumerate}
                                \item $EV=\sum_{N=1}^N\frac{FCF}{(1+WACC)^N}+\frac{TV}{(1+WACC)^{(N+1)\longrightarrow\infty}}$
                        \end{enumerate}
                \end{enumerate}
                \item Value of Synergies
                \begin{enumerate}
                        \item Calculate stand-alone value of firms ($EV_i+EV_j$) (this yields $V_{Combined\ w/o\ S}$)
                        \item Calculate combined value of firms ($EV$) using re-computed values for the combined firm in sections 1 and 2 of this tutor (this yields $V_{Combined\ w/\ S}$)
                        \item Calculate the value of the synergy (sections (b)$-$(a) in this subsection of the tutor) ($V_S=V_{Combined\ w/\ S}-V_{Combined\ w/o\ S}$)
                \end{enumerate}
	\end{enumerate}

        The table below shows an example of the steps listed above, using sample values which meet the criteria discussed formerly. Note that as $\left(\frac{Debt}{Equity}\right)_{i}\approx\left(\frac{Debt}{Equity}\right)_{j}$, debt-to-capital$_{i}$ $\approx$ debt-to-capital$_j$ as debt-to-capital $=$ $\frac{Debt}{Debt+Equity}$.

        \vspace\fill

        \begin{center}
                \textsc{Turn Page For Example}
        \end{center}

        \newpage

	\begin{center}
                \scriptsize
                %\renewcommand{\arraystretch}{1.25}
        \begin{longtable}{p{6.5cm}|p{1.5cm}p{1.5cm}p{1.5cm}}
                & Firm I & Firm II & Combined\\
                \hline
                \hline
                \multicolumn{4}{c}{Inputs}\\
                \hline
                \hline
                Risk-Free Rate ($R_f$) (E.g. T-Bill$_{\textrm{10yr}}$) & \multicolumn{3}{c}{2.226\%}\\
                Market Risk Premium ($R_m-R_f$) & \multicolumn{3}{c}{3.110\%}\\
                Beta Value ($\beta$) & 1.25 & 1.25 & 1.25\\
                \hline
                Pre-Tax Cost of Debt ($r_D$) & 4.00\% & 4.00\% & 4.00\%\\
                Tax Rate ($T$) & 35.00\% & 35.00\% & 35.00\%\\
                Debt-To-Capital Ratio ($w_D=\frac{Debt}{(Debt+Equity)}$) & 23.70\% & 23.70\% & 23.70\%\\
                \hline
                Revenue & 696.00m & 226.00m & 922.00m\\
                EBIT & 83.00m & 46.00m & 129.00m\\
                \hline
                Pre-Tax Return on Capital (ROC) & 8.30\% & 5.00\% & 7.20\%\\
                Reinvestment Rate & 75.00\% & 75.00\% & 75.00\%\\
                Length of Growth Period & 5 & 5 & 5\\
                \hline
                \hline
                \multicolumn{4}{c}{Outputs}\\
                \hline
                \hline
                \textit{Firm Risk Premium} ($\beta(R_m-R_f)$) & 3.89\% & 3.89\% & 3.89\%\\
                \hline
                Post-Tax Cost of Debt ($r_{D_{\textrm{Post-Tax}}}=r_D(1-T)$) & 2.6\% & 2.6\% & 2.6\%\\
                Cost of Capital$^1$ & 5.3\% & 5.3\% & 5.3\%\\
                \hline
                Post-Tax Return on Capital (ROC) & 5.39\% & 3.25\% & 4.7\%\\
                Expected Growth Rate ($g^e$) for $t\in\{1,...,5\}$ & 4.00\% & 2.4\% & 3.5\%\\
                \hline
                \hline
                \multicolumn{4}{c}{Value of Firms}\\
                \hline
                \hline
                PV of FCF & 65.00 & 34.00 & 100.00\\
                Terminal Value & 1273.00 & 653.00 & 1928.00\\
                Enterprise Value & 1050.00 & 539 & 1590\\
                \hline
                \hline
                \multicolumn{4}{c}{Value of Synergies}\\
                \hline
                \hline
                Value of Firms (Stand-Alone) & \multicolumn{3}{c}{1588.90}\\
                Value of Firms (Combined) & \multicolumn{3}{c}{1590.11}\\
                Value of Synergies & \multicolumn{3}{c}{1.20}\\
		\hline
	\end{longtable}
        \end{center}

	\vspace\fill

	\newpage

	\subsection{Question 6: Cross-Border Mergers}

	\textit{What are the main challenges facing cross-border mergers and acquisitions? Illustrate some of these with case studies discussed in the course. What is the basic principal underlying the European directive on cross-border mergers and what are its implications?}\\

	Piss off.

\newpage

\section{April 2019 Paper} 

	\subsection{Question 1: Merger Theory}

	\textit{Discuss the Coase framework for the size of the firm and relate it to the value enhancing theory of mergers. With reference to the literature, discuss the empirical evidence on combined merger returns.}

		\subsubsection*{Background}

	\newpage

	\subsection{Question 2: Merger Firm Value}

	\textit{
		\begin{itemize}
		\setlength\itemsep{0cm}
			\item[$a)$] Briefly discuss the comparable transaction approach for valuing companies in the context of a merger. [30\%]
			\item[$b)$] Briefly outline the steps involved in estimating the synergy of an acquisition. Assume that the firms have the same pre-tax cost of debt, the same debt-to-value ratio, the merged entity will have the same debt-to-value ratio and the tax rates for the merging and merged firms are all the same. Assume an initial growth phase of high growth driven by high reinvestment rates followed by a terminal phase of lower growth. [70\%]
		\end{itemize}
	}

		\subsubsection*{Background (a)}
	
	Valuing firms can be difficult in the context of a merger. Depending on the firm of type of person relevant, there may be different logical and philosophical views taken regarding how the value of a firm should be observed. Generally, we refer to two primary areas when discussing this; comparables, such as comparing transactions, and discounted cash flow, which accounts for many of the algebraic numberic methods used to generate estimations in capital budgeting, capital structure etc. Of course, these methods are usd to accurately forecast and target mergers and acquisitions, and aid in decision making to increase the subsequent value of a merged firm.

		\subsubsection*{Logic (a)}

	The comparable transaction approach is most commanly a traditional and logical approach to valuing and comparing firms which are being targeted for acquisition. If the targets of the acquisition has a similar business model to the acquirer, this portocol is relevant. This means that when evaluating a firm, their clients, products, size of transactions, operations, capital structure, trends and prospects, etc., must be somewhat relevant. According ratios are generated for the evaluated firms as a result. This approach is often referred to as the `common sense' approach as it uses accurate and fair market data and trends to generate decision making influence. This extends to the idea that amateur analysts and experienced often gather the relevant data from similar places. That is, public releases of the targets, databases such as Bloomberg's, CRPS etc., 8-K reports, and research and development notes of targets.\\

	If valuers seek the most accurate valuation of a firm, they must account for contexts in which there are [1] transactions from very similar, if not identical assets, of firms; [2] transactions closer to the data of valuation, simply to provide an accurate and up-to-date representation of the market; [3] transactions from assets which act in similar capacity and are approximately at the same level of `tangibility' across firms; [4] sufficient and similar types of information surrounding assets and transactions across firms, which originate from credible sources; and [5] ensuring transactions are real and not `intended'.

		\subsubsection*{Ratios \& Comparables (a)}
	
	We tend to observe ratios which are traditional indicators of success. These often inclue enterpise-value-to-sales, enterprise-value-to-EBITDA, price-to-earnings, paid-to-salespaid-to-book, paid-to-net-income, etc. Across relevant firms, these ratios are averaged to produce a \textit{transaction multiple} which are given as the coefficient to the according values from a single target firm; generating a `normalized' equity value. These are then averaged to produce an `accurate' comparable equity value. An example is displayed as follows:
	
	\begin{table}[h]
		\scriptsize
		\renewcommand{\arraystretch}{1.25}
	\begin{center}
	\begin{tabular}{p{4cm}|p{2cm}p{2cm}p{2cm}|p{2cm}}
		\multirow{2}{*}{\textbf{Ratio}} & \multicolumn{4}{c}{\textbf{Firm \& Comparative Value}}\\
		\cline{2-5}
		& \textbf{Amoco} & \textbf{Texaco} & \textbf{Conoco} & \textbf{Average}\\
		\hline
		Total Paid : Sales & 1.38 & 0.77 & 0.37 & 0.84\\
		Total Paid : Book & 3.00 & 2.79 & 2.29 & 2.69\\
		Total Paid : Net Income & 22.46 & 15.46 & 7.60 & 15.18\\
		Premium Paid, \% Target & 22.3\% & 17.7\% & 0.0\% & 13.3\%\\
		Premium Paid, \% Combined & 7.7\% & 6.3\% & 0.0\% & 4.7\%\\
		\hline
	\end{tabular}
		\caption{Sample Comparable Transactions}
	\end{center}
	\end{table}

	\begin{table}[h]
                \scriptsize
                \renewcommand{\arraystretch}{1.25}
        \begin{center}
	\begin{tabular}{p{3.5cm}|p{2cm}p{4.5cm}|p{2cm}}
		\multirow{2}{*}{\textbf{Component}} & \multicolumn{3}{c}{\textbf{Firm \& Comparative Value}}\\
		\cline{2-4}
		& \textbf{Mobil} & \textbf{Average Transaction Multiple} & \textbf{Equity Value}\\
		\hline
		Past 12-Month Sales & \$63.0 & 0.84 & \$53.0\\
		Book Value & \$19.0 & 2.69 & \$51.2\\
		Past 12-Month Net Income & \$2.9 & 15.18 & \$43.8\\
		Market Value Target & \$58.7 & 13.3\% & \$66.5\\
		Market Value Combined & \$233.7 & 4.7\% & \$69.6\\
		\cline{4-4}
		Average & \multicolumn{2}{c|}{} & \$56.8\\
		\hline
	\end{tabular}
                \caption{Sample Average Transaction Multiples}
        \end{center}
        \end{table}

		\subsubsection*{Positives \& Negatives}

	Conclusions from this method are based on freely available public information, meaning they should be consistently reliable and credible. It can provide a good reference of industry-level multiplpes at various period in market cycles. The method can also highlight the types of transactions which are relevant to current firms and buyers, and their asset requirements. However, this emthod can sometimes be na\"{i}ve in the sense that it may not account for varying conditions under which some transactions may have occured. It also doesn't focus on the value of potential synergies and other relevant intangibles. Thus, it's argued that these comparables frequently don't account for growth of revenue, risk, stage of a firm in its cycle, competition, and expansion etc.

		\subsubsection*{Background (b)}

	Valuing synergies is a crucial component of establishing the value of a merger as essentially, the merger's future value is based on how the adjoined firm will use synergies to enhance the growth rate of current operations and assets. Based on the types of synergies proposed in the merger, the value of the merged firm may change over time depending on: an daltered revenue growth rates from \textit{growth synergies}; altered margins due to synergies related to \textit{economies of scale and scope} etc.; altered tax rates due to tax beneits proposed in \textit{tax synergies}; altered costs of financing (debt and equity) due to \textit{financing synergies}; altered debt ratio due to change in risk attitude. Generating these factors on top of the proposed combined firm yields the total value of the combined firm with synergies. To observe the value of the synergies, the value of the (to-be) acquired firm and its control premium, and the value of the (to-be) acquirer are to be removed from the total value. Note that a control premium accounts for the value of a firm if it were to be optimized from the sense of investment, financing and dividend policies/decisions.

		\subsubsection*{Total Value of Combined Firm: Outline (b)}

	As noted, the first step is to value the entirity of the combined firm with the proposed synergies integrated. When doing this, we primarily look at \textit{operating synergies} and \textit{financial synergies}. These vary depending on the nature of the merger. For example, a horizontal merger is likely to see increased economies of scale, increased market power thus, reduced costs and inproved margins. Vertical mergers mostly see operational synergies form efficiencies in production chains and enhanced scope. So forth, firms must consider [1] where imporved margins come from (i.e. dominantly reduced costs or raised revenues); [2] if there is grwoth potential to the synergies or if they are more immediately functional; and [3] when the costs and the gains synergies will begin to take effect. This could also relate to the former two as expected expenditures and revenues may be effected by the ability produce synergy realization fast. Thus, from this and the formerly discussed, for ..., we know:

	$$V_S=V_{Combined\ w/\ S}-V_{Combined\ w/o\ S}$$
	$$\therefore V_S=V_{Combined\ w/\ S}-(V_{i_{Pre-Merger}}+V_{j_{Pre-Merger}})$$

		\subsubsection*{Total Value of Combined Firm: Steps (b)}

	Recall that in this scenario, firms should have equivalent pre-tax cost of debt, debt-to-value ratio, and this debt-to-value ratio holds for the merged firm. Tax rates remain the same. There's initial high growth stimulated by reinvestment followed by lower growth. Steps of valuation are outlined below.
	
	\begin{enumerate}
	\setlength\itemsep{0cm}
		\item Inputs
		\begin{enumerate}
			\item Declare risk-free rate ($R_f$) and market risk premium ($R_m-R_f$)
			\item Declare beta values ($\beta$) for each firm
			\item Declare risk premium for firms ($\beta(R_m-R_f)$)
			\item Declare pre-tax cost of debt ($r_D$), tax rate ($T$), debt-to-captial ratio ($w_D=\frac{Debt}{Debt+Equity}$), equity-to-capital ratio ($w_E=\frac{Equity}{Debt+Equity}$)
			\item Declare revenues, EBIT; calculate collective revenue, EBIT
			\item Declare pre-tax return in capital ($ROC_{\textrm{Pre-Tax}}$), reinvestment rate, length of initial high growth period ($N$ for time $t\in\{1,...,N\}$)
		\end{enumerate}
		\item Outputs
		\begin{enumerate}
			\item Calculate cost of equity ($r_E=(R_f+\beta)(R_m-R_f)$)
			\item Calculate post-tax cost of debt ($r_{D_{\textrm{Post-Tax}}}=r_D(1-T)$)
			\item Calculate cost of capital
			\begin{enumerate}
				\item $WACC=r_Dw_D(1-T)+r_Ew_E$
				\item $\therefore WACC=r_D\left(\frac{Debt}{Debt+Equity}\right)(1-T)+r_E\left(\frac{Equity}{Debt+Equity}\right)$
			\end{enumerate}
			\item Calculate post-tax return on capital
			\begin{enumerate}
				\item $ROC_{\textrm{Post-Tax}}=ROC_{\textrm{Pre-Tax}}(1-T)$
			\end{enumerate}
			\item Calculate expected growth rate for period $N$ of high growth
			\begin{enumerate}
				\item $g^e=ROC_{\textrm{Post-Tax}}(Reinvestment\ Rate)$
			\end{enumerate}
		\end{enumerate}
		\item Value of Firms
		\begin{enumerate}
			\item Calculate the present value of free cash flows (FCF) for period of high growth 
			\begin{enumerate}
				\item $PV_{FCF}=\sum_{N=1}^N(EBIT(1-T)(1-Reinvestment\ Rate))((1+g^e))\left(\frac{\frac{1-(1+g^e)^N}{(1-WACC)^N}}{WACC-g^e}\right)$
			\end{enumerate}
			\item Calculate terminal value
			\begin{enumerate}
				\item $TV=\sum_{N=1}^N(EBIT(1-T))((1+g^e)^N)(1+R_f)\left(\frac{\frac{1-R_f}{WACC}}{WACC-R_f}\right)$
			\end{enumerate}
			\item Calculate enterprise value
			\begin{enumerate}
				\item $EV=\sum_{N=1}^N\frac{FCF}{(1+WACC)^N}+\frac{TV}{(1+WACC)^{(N+1)\longrightarrow\infty}}$
			\end{enumerate}
		\end{enumerate}
		\item Value of Synergies
		\begin{enumerate}
			\item Calculate stand-alone value of firms ($EV_i+EV_j$) (this yields $V_{Combined\ w/o\ S}$)
			\item Calculate combined value of firms ($EV$) using re-computed values for the combined firm in sections 1 and 2 of this tutor (this yields $V_{Combined\ w/\ S}$)
			\item Calculate the value of the synergy (sections (b)$-$(a) in this subsection of the tutor) ($V_S=V_{Combined\ w/\ S}-V_{Combined\ w/o\ S}$)
		\end{enumerate}
	\end{enumerate}

	The table below shows an example of the steps listed above, using sample values which meet the criteria discussed formerly. Note that as $\left(\frac{Debt}{Equity}\right)_{i}\approx\left(\frac{Debt}{Equity}\right)_{j}$, debt-to-capital$_{i}$ $\approx$ debt-to-capital$_j$ as debt-to-capital $=$ $\frac{Debt}{Debt+Equity}$.

	\vspace\fill

	\begin{center}
		\textsc{Turn Page For Example}
	\end{center}

	\newpage

	\begin{center}
		\scriptsize
		%\renewcommand{\arraystretch}{1.25}
	\begin{longtable}{p{6.5cm}|p{1.5cm}p{1.5cm}p{1.5cm}}
		& Firm I & Firm II & Combined\\
		\hline
		\hline
		\multicolumn{4}{c}{Inputs}\\
		\hline
		\hline
		Risk-Free Rate ($R_f$) (E.g. T-Bill$_{\textrm{10yr}}$) & \multicolumn{3}{c}{2.226\%}\\
		Market Risk Premium ($R_m-R_f$) & \multicolumn{3}{c}{3.110\%}\\
		Beta Value ($\beta$) & 1.25 & 1.25 & 1.25\\
		\hline
		Pre-Tax Cost of Debt ($r_D$) & 4.00\% & 4.00\% & 4.00\%\\
		Tax Rate ($T$) & 35.00\% & 35.00\% & 35.00\%\\
		Debt-To-Capital Ratio ($w_D=\frac{Debt}{(Debt+Equity)}$) & 23.70\% & 23.70\% & 23.70\%\\
		\hline
		Revenue & 696.00m & 226.00m & 922.00m\\
		EBIT & 83.00m & 46.00m & 129.00m\\
		\hline
		Pre-Tax Return on Capital (ROC) & 8.30\% & 5.00\% & 7.20\%\\
		Reinvestment Rate & 75.00\% & 75.00\% & 75.00\%\\
		Length of Growth Period & 5 & 5 & 5\\
		\hline
		\hline
		\multicolumn{4}{c}{Outputs}\\
		\hline
		\hline
		\textit{Firm Risk Premium} ($\beta(R_m-R_f)$) & 3.89\% & 3.89\% & 3.89\%\\
		\hline
		Post-Tax Cost of Debt ($r_{D_{\textrm{Post-Tax}}}=r_D(1-T)$) & 2.6\% & 2.6\% & 2.6\%\\
		Cost of Capital$^1$ & 5.3\% & 5.3\% & 5.3\%\\
		\hline
		Post-Tax Return on Capital (ROC) & 5.39\% & 3.25\% & 4.7\%\\
		Expected Growth Rate ($g^e$) for $t\in\{1,...,5\}$ & 4.00\% & 2.4\% & 3.5\%\\
		\hline
		\hline
		\multicolumn{4}{c}{Value of Firms}\\
		\hline
		\hline
		PV of FCF & 65.00 & 34.00 & 100.00\\
		Terminal Value & 1273.00 & 653.00 & 1928.00\\
		Enterprise Value & 1050.00 & 539 & 1590\\
		\hline
		\hline
		\multicolumn{4}{c}{Value of Synergies}\\
		\hline
		\hline
		Value of Firms (Stand-Alone) & \multicolumn{3}{c}{1588.90}\\
		Value of Firms (Combined) & \multicolumn{3}{c}{1590.11}\\
		Value of Synergies & \multicolumn{3}{c}{1.20}\\
		\hline
	\end{longtable}
	\end{center}

	\newpage

	\subsection{Question 3: Merger Waves}

	\textit{Compare and contrast the merger wave of the 1960’s with that of the 1980’s. Briefly discuss the exogenous factors that tend to influence mergers and could also lead to merger waves, with reference to the literature in this area.}\\

	\textit{This question has been extended to compare and contrast all merger wave periods, as opposed to those listed in the above question.}

                \subsubsection*{Background}

        It is argued that there are five observed `merger waves', with an arguable sixth and seventh in more recent times. `Merger waves' are defined as a temporary period in which an industry or market etc. experiences substancially increased merger volume. Over these periods, merger waves have generally presented and/or been present within a set of six primary characteristics. These include: periods of high economic growth, contexts of favourable stock prices, periods of technological change, input price volatility, periods of legal and regulatory change, and innovations in financing and accounting methodologies.

                \subsubsection*{Wave of 1895--1904}

        This first wave of high merger volume was observed at the first major progression in infrastructure, production and manufacturing etc. There was a greater demand for the transportation of goods and people across borders so this called for increased rail infrastructure. Alongside this was the compliemntary advancements of electrical energy and its disribution of use across industries. The distribution of oil, metal ore, other ores, food goods etc., was increasing immensely. So fourth, this period called for great economies of scale to deal with the demand and needs. This was a period in which brands, people and firms were defining themselves and their specialities so, desired large market reach. Hence, a monopolistic driver through horizontal mergers. Evident from this, the relelvant industry/market characteristic shown in this period reflected a period of high economic growth and technological change. Lipton (2006) argues that `The Panics' of 1904 and 1907 and The Great War were the primary factors which had input in putting an end to this wave. In 1904, the Supreme Court made the decision to make amended antitrust laws applicable to horizontal mergers, in effect making mergers of this period and of these characteristics less desirable.

                \subsubsection*{Wave of 1922--1929}

        The merger wave of 1922--1929 was a a period primarily made up of product extention. After various passings of new laws and regulation in reaction to the first wave, combatting monopolistic tendencies etc., this period saw many vertical mergers in an attempt to combine efficiencies of industries such as mining, ifrastructure, building etc. Thus, an oligopolistic period. There was a huge demand by firms for scope for mass distribution. Lipton (2006) observes and argues that this is the period in which automobile and other mobile manufacturers made a name for themselves. For example, Ford became fully integrated with its supply and distribution; steel mills, railroads, factories, docks and shipyards etc. Thus overall, this period observed reaction to a period of economic growth and legal and regulatory changes.

                \subsubsection*{Wave of 1960's}

	The 1960's was a period dominated by conglomorate mergers. Lipton (2006) argues this period of high volume was apparent between 1955 and its floating end of 1969--1973, however it's not completely certain exactly how long this period lasted. It is known as the period where the market was ‘rewarding’ diversification. There was a large opportunity at this point for large established firms to being acquiring other firms, not for vertical reasons discussed prior, but simply for more basic financial and economic reasons. Firms aimed to reduce instability by diversifying their scope of operations. This was especially apparent in low-growth-prospect markets. Companies aimed to satisfy wide post-World War II demands by broadening their offerings. This not only meant more revenue form more places but, a safety net in reaction to future changing demands as more businesses could support this.\\

	Convertable bonds also played a role. Frequently two firms with low growth prospects merged to imporve P/E and EPS, by focussing on decreasing the denominator of the latter. This was accomplished by is- suing a substantial amount of convertable bonds which would not be classed as shares. This was adided by the driving of earnings by the merger also. However, various legal acts, including the Tax Reform Act of 1969, which put an end to manipulation of convertable bonds by re- quiring their accountance as if converted. Additionally, laws were passed regarding conglomerate logistics also. Evidently, the driving characteristic in this wave was innovation of financing methods. Lipton (2006) finds that a stock crash in these diversified conglomerates started in 1969 and lasted through to the early 70's, seeing the end of this wave. They also suggest that many of the firms did not fully realise the full potential of their diversification due to lacking liquidity and investment post-crash.

                \subsubsection*{Wave of 1981--1989}

        This period is known as the unwinding of the conglomerate and diversification wave. It is also in the context of response to low economic growth in the 1970's, meaning motives for this wave began to form in the mid-70's. However, Lipton (2006) argues that its volumetric scale took formed from 1984--1989. This is convenient as it echos the immense passion, drive, motivation and success of the iconic 1984--1989 television masterpiece; a Michael Mann classic, the stomping ground of Jan Hammer, the birthplace of Don Johnson and Philip Michael Thomas, the home of the long, cold, intimidating stares of Lieutenant Castillo; \textit{Miami Vice}.\\

        This wave was more of a reactive period than a proactive one; many of the 60's' congolerates had failed in the sense that the mergers had worrked in an inverse manner to that sought. In result, the sum of the individual parts of the merged firms were was greater than thier whole. Therefore, the reaction was to unwind this and operate firms in their stand-alone form. There was also large economic growth around this period. One part of this was the issuance of junk bonds, where firms with cash-raising problems issue bonds with low credit ratings and promise for high yield. These were designed to be bought in bulk as part of a `diverse' package. This attracted buyers and therefore raised cash, in a completely new market. It raised more capital than expected which went towards acquisitions. Many of these were `bustup' acquisitions; focusing the breaking up firms with part-sums greater than their whole. Pieces were sold off and revenue used to reduce debt.\\

        Much of the reaction in this period was also related to hostile takeovers. Lipton (2006) argues that this aspect reaches back to 1974 when the first `major firm' hostile bid was made by Morgan Stanley, on behalf of Inco. It's said that this then opened the floodgates to investment banks' making hostile bids and successful hostile takeovers. This hostile nature wasn't just the case in financial institutions, however. It was present in many manufacturing contexts for example, with the iconic \textit{Dornaus \& Dixon Bren Ten} 10mm police and government issue handgun/sidearm; where a financially and logistically suitable solution to the issue they faced with the subcintractor who supplied the gun's magazines was to forceably acquire they operations. This provided fast return to market. This was also the handgun used by Detective James ``Sonny'' Crockett so of course, they had to maintain their image and reuptation as a reliable firearm manufacturer.\\

	Primarily, this wave was driven by characteristics including high economic growth, legal and regulatory change and further financing innovations. Lipton (2006) argues that it began to come to a close in 1987 after the stock market crash (echoing the declined quiality in Dick Wolf's takeover from Michael Mann as Executive Producer of season 4 of \textit{Miami Vice} in 1897). It ended with the \$25 billion RJR Nabisco LBO and the fall of the junk bond market, in 1989 (echoed in the emotional decelleration towards the end of season 5 of \textit{Miami Vice} and the weak knees which came with the final hearing of \textit{Crockett's Theme} during that iconic hand shake in the 2-hour final\'{e}, \textit{Freefall}). This decline is said to be attributed also to the collapse of savings and loan banks and the capital problems faced by commercial banks.

                \subsubsection*{Wave of 1992--2000}

        The merger wave of the 1992--2000 period saw the innovation and rise of brand new firms and even industries/subindustries. For example, following the huge success with PS2 Terminals and Model M’s in the 80’s/90’s, IBM began taking over every workplace with the start of their famous T-Series and X-Series of portable computers (ThinkPads), UltraBases, UltraDocks and business accessories for the pragmatic businessman. This was not only a new opportunity for many industries in manufacturing, engineering and computer science; it was also a huge one simply because of the percieved mass `need' for these business items, across most business platforms. Therefore, this was a period of P/E and EPS increae (attrac- tiveness to investors and acquirers etc.) driven by the numerator of the latter; earnings. This expansion wasnt only happening in relative domestic markets, globalization was rising; increasing requirement for scale in man- ufactuing and distribution etc. thus, mergers. A great example of network exapansion was the increase in demand for the AT\&T Merlin telecommunications system. This was a growth and earnings-driven period, not debt.\\

        This period is referred to by Lipton (2006) as the `mega-deal era'. Firms held the belief in this context that `size matters'. This saw already-established, large firms become even more bullish. There was a preception that to continue to trade on an international scale, firms had to be of great size. In many cases, this led to the merging of firms which traditionally would not seem to fit together. Its oberved that these unalike mergers were not always carried out on purely fundamental grounds, for example, like in the rational reasoning of the joining of unalike firms in many vertical mergers. This may bot even have just been because firms were unalike however, there may simply just not have been a need for, or any substantial synergies in, the mergers. Lipton (2006) argues that the mergers of Citibank and Travelers, Chrysler and Daimler Benz, Exxon and Mobil, Boeing and McDonnell Douglas, AOL and Time Warner, and Vodafone and Mannesmann, etc., fit into this categorization.\\

        Merger volume rose from \$342 billion in 1992 to \$3.3 trillion in 2000, with 90\% of the largest mergers taking place from 1998--2000. The AOL and Time Warner merger holds the record at a value of \$165 billion. Mergers during this period cleanrly saw signs of motivation from technological change and favourable stock prices. Lipton (2006) claims its end came with the burst of the Millenium Bubble after scams and scandals such as those of Enron. In many cases, it's argued that lessons learned during the closing period of this wave provided much of the knowledge firms now apply to practice in corporate governance etc. The sad decelleration of telecommunications and technological business saw a large decline in tech merger volume and in effect, saw a fall of \~{}50\% in the NASDAQ from its 90's high. Subsequently, the junk bond market became close to non-existent, many lenders constricted requirements and standards, and mergers were generally not observed as very positive things for a while after this point.

		\subsubsection*{Rationale: Exogenous Factors}

        Lipton (2006) also describes various exogenous factors (factors which are determined by industries, markets, economic states etc., on a larger scale) driving mergers.

        Pooling accounting is one of the biggest directly financial motives in mergers. In the 1990's it helped avoid traditionally necessary `goodwill amortization' thus, avoiding particular dilution of earnings. This was especially apparent when an acquired firm was accounted for as a purchased entity. From 2001, `purchase accounting' replaced this idea of pooling. which further amends and removes burdens associated with share repurchases and asset dispositions, post-merger.\\

        Furthermore, activist hedge funds and activist institutional investors can significantly influence how a firm operates their sales and acquisitions of business. Simply, they have the ability to exert much force on the board of directors if they, sometimes na\"{i}vely, belive otherwise to directors regarding the use, efficiency or relevance of components of operations.\\

        Traditionally, various governmental policies have held the ability to overpower, prohibit or delay mergers. In the mighty 1980, The Shermen Act was passed in Congress to ``preserve free trade and competition''. In addition to the Federal Trade Commission Act and the Clayton Act, the set of US Antitrust laws is formed. These allow and allocate the correct licences and freedoms and restrictions in order to allow firms to act correctly in merger activity.\\

        Frequently arbitrageurs, hedge fund activists, and activist institutional investors combine forces in order to encourage firms to seek mergers. The often provide liquidity required for shares of firms associated with mergers; buying up large amounts of stock.\\

        Currency fluctuations very frequently influence cross-border mergers. Firms which exist in countries with strong currencies tend to have greater effect in acquisitions than firms in countries with poor currencies. For example, the decline of the Euro in 2000 saw huge acquisition of European firms by US firms. And, the same again with a strong dollar and US firms' acquisition of Asian firms.\\

        Deregulation has aided the global movement towards capitalism and privitization of firms which over time, has led to an increased pool of eligible candidated for mergers. Otherwise, protectionism tends to prohibit mergers. Some, perhaps more in-touch with traditional values, of us view this as a positive as it is what protects the fundamental value and function of what was once good.\\

        The increase of merger `experts' in the field has also aided analysis, conception, valuation and execution of successful mergers due to greater focus. This often provides firms with the confidence they need to interface with a merger and its componenets; encouraging greater merger volume. This is generally done by bodies associated with investment banks. This is also aided by specialist lawyers and consultants etc., in the field.\\

        With reference to the first and second `merger waves', brand-new firms types were appearning on the market. New technology firms and firms alike were often seeking temporary growth then to be acquired. The present days sees that if a new-looking and innovative start-up is formed, frequently larger firms look to acquire these to `help them realise potential'. Well, to claim the benefits from their growth.

	\newpage

	\subsection{Question 4: LBO Operation}

	\textit{What are the types of target industries and target firms involved in leveraged buyouts? Describe the value stages of a typical LBO operation and discuss the empirical results relating to the magnitude of gains as well as the takeover premiums for LBO’s in the 1980’s. Briefly discuss the sources of gains in an LBO.}\\

	Piss off.

	\newpage

	\subsection{Question 5: Payment \& Bidders}

	\textit{
		\begin{itemize}
                \setlength\itemsep{0cm}
                        \item[$a)$] Briefly discuss the empirical findings relating to how the method of payment and the number of bidders affects bidder returns in a merger. [35\%]
			\item[$b)$] Why is the usual method of calculating the weighted average cost of capital unsuitable for a leveraged buyout? Briefly outline the capital cash flow method for valuating a leveraged buyout and indicate how it is different from the usual DCF valuation for a firm. [65\%]
		\end{itemize}
	}

	Piss off.

	\newpage

	\subsection{Question 6: Cross-Border Mergers}

	\textit{Briefly discuss the main factors that determine the success or failure of cross- border mergers and acquisitions. Illustrate some of these with case studies discussed in the course. What is the basic principal underlying the European directive on cross-border mergers and what are its implications?}\\

	Piss off.

\end{document}
