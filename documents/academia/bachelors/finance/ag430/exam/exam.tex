\documentclass[11pt, english]{article}              
        \usepackage{geometry}
                \geometry{                          
                        a4paper,total={210mm,297mm},
                        tmargin=40.8mm,
                        bmargin=40.8mm,
                        lmargin=32.6mm,
                        rmargin=32.6mm,
                }

        \usepackage{titlesec}         
                \titleformat{\section}
                        {\normalfont\fontsize{18}{16}\bfseries}{\thesection}{0.5em}{}
                \titleformat{\subsection}
                        {\normalfont\fontsize{14}{16}\bfseries}{\thesubsection}{1em}{}
                \titleformat{\subsubsection}
                        {\normalfont\fontsize{11}{16}\bfseries}{\thesubsubsection}{1em}{}

        \usepackage{longtable}
        \usepackage{multirow}

        \usepackage[labelfont=bf,textfont=bf,font=small,skip=8pt]{caption}

        \setlength{\parindent}{0pt}
        \renewcommand{\baselinestretch}{1.25}
       	\usepackage{setspace}

        \usepackage{amsmath}
        \usepackage{amssymb}
        
        \usepackage{graphicx}

\begin{document}

\pagenumbering{gobble}

        \title{\textsc{AG430 Corporate Financing\\ Coursework Examination}}
        \author{\textsc{Lewis Britton}}
        \date{\textsc{Academic Year 2020/2021}}
        \maketitle

\newpage

\pagenumbering{roman}

        \renewcommand{\contentsname}{Table of Contents} 

	\tableofcontents

\newpage

\pagenumbering{arabic}

\section{Section 1: Payout Policy}

	\subsection{Question 1: Payout Policy \& Signalling}

	\textit{Fama and French (2001) report that the number and proportion of publicly listed US industrial firms paying a dividend declined significantly between 1978 and 1999. Critically evaluate the various explanations for this phenomenon using appropriate studies of corporate payout policy. Your evaluation should consider the evidence from financial markets in the United States, as well as international markets.}

		\subsubsection*{Background}

	The `disappearing dividend phenomenon', also simply decreasing dividends, occurs when a firm ether stops paying dividends or loses the ability to do so. Losing the ability may stem from delisting or mergers etc.\\

	The primary study surrounding this issue is Fama and French's (2001) investigation of the characteristics of firms who pay and firms who do not pay dividends. They state that, in the case of this study, they consider `non-payers' as firms who simply do not pay dividends and `payers' as: [1] new firms making an entry to dividend payments in their year of listing, [2] firms which are choosing to begin paying a dividend and, [3] firms resuming dividend payments. On the surface, Fama and French (2001) find that there is a decrease of 45.7\% in firms paying dividends over the period of 1978-1999. This study takes place over the NYSE, AMEX and NASDAQ. They outline findings that a firm with high profitability, great size, and weak investment opportunities are likely to be dividend payers. Inferred from this is low profitability, little size, and strong investment opportunity firms are likely non-payers. Table 1, below, outlines this in a more efficient manner. It is therefore assumed that these firms are the most evident in the `disappearing dividend phenomenon'.\\

	\begin{table}[h]
		\scriptsize
	\begin{center}
	\begin{tabular}{p{3.5cm}p{2.5cm}p{2.5cm}p{2.5cm}}
		\textbf{Firm Cat.} & \textbf{Profitability} & \textbf{Growth} & \textbf{Investment Opp.}\\
		\hline
		Dividend Payers & $\uparrow$ & $\uparrow$ & $\downarrow$\\
		Dividend Non-Payers & $\downarrow$ & $\downarrow$ & $\uparrow$\\
		\hline
	\end{tabular}
		\caption{Payer \& Non-Payer Characteristics}
	\end{center}
	\end{table}

		\newpage

		\subsubsection*{Evidential Findings}

	Fama and French (2001) highlight the fact that, aligning with expectations, firms who have never paid a dividend have the greatest investment opportunities. This is due to these firms' focus on research and development and growing their operations. Firms like this tend not to value actual financial growth as much due to the fact that they are frequently run by entrepreneurs who care more for making innovations, developments and breakthroughs in an industry. In this sense, it is important that they purpose cash towards areas, such as R\&D, which will allow them to arrive first at a goal. They may not have any established competition in their particular branch of their industry, in their proposed innovations, so that is why being first is important. These firms are therefore frequently financially smaller due to their goals. This is reinforced where Fama and French (2001) find that dividend payers, in their index sample, account for 93.5\% of the aggregate book-value-of-assets, between 1973-1977.\\

	Furthermore, Fama and French (2001) find that overall firm inclination to pay dividends, regardless of firm categorization/characteristics, is lower than expected. They find a declining propensity to pay dividends over the period of 1978-1998. They offer reasoning for this relating to the act that post-ate-90s, firms began to consider tax advantages of not paying dividends. However following this research, DeAngelo et al. (2004) found that from 1978-2000 the real number of dividends paid increased by 22.7\%. This suggests that the overall number of firms who pay dividends did in fact decrease, in-line with the findings of Fama and French. However, many firms began to increase dividend payments and, in effect with increased investment, increase number of dividends paid as a firm. More specifically, this increased number of dividends paid in a firm was found, in-line with expectations and general reasoning behind dividends, to increase with increased earnings.

		\subsubsection*{A Different Perspective}

	Grullon and Michaely (2002) approach the issue from a slightly different point of view. They explore more into the reasoning behind the number of firms paying dividends decreasing. Its stated that may firms are moving to a repurchase- oriented form of payout policy, as opposed to dividend payments. Evidentially, they showed the decrease in firms paying dividends increased from 1980-2000, from 4.8\% to 41.8\%. In these cases, firms are found to be replacing once-dividend payouts with repurchases. In-line with a former argument regarding number of firms and number of dividends/value paid per firm, Grullon and Michaely (2002) found that firms were still paying high numbers and values of dividends but, the number of firms favouring dividends was decreasing; in favour of the number of firms favouring repurchases rising. Furthermore, the same study found that firms which are new to listing/payout policy or, haven't paid dividends in the past are becoming more likely to engage in repurchases as part of their policy. This is shown in an estimated 0.0243 probability that the discussed firms would pay dividends and an estimated 0.2962 probability that they would engage in repurchases.\\

	Kirkulak and Kurt (2010) support prior arguments from the perspective of low- earning firms paying no dividends or paying smaller-value dividends. They find evidence of this in the context of financial crises and the subsequent lack of ability in firms to make consistent or increasing earnings. This could also be linked to debt in cases of financial distress, although there is not much evidence.

		\subsubsection*{Contradictions}

	Michaely and Moin (2020) investigate the appearance of `re-appearing dividends', where dividend numbers began to increase again. They found that between 2000- 2018, dividends increased by \~{}36\%. However, it is probable that this increase is in-line with an decrease in small (low profitability/growth) firms becoming more potent meaning there is a larger population of firms who generally opt not to pay dividends. Many firms who were once payers but opted out (decreasing aggregate dividends) but have been found to delist because of shrinking growth, meaning the listed population becomes more dominant in favour of payers. This supports previous arguments of the relative sort.

		\subsubsection*{Conclusions}

	Overall, many studies conclude with evidence similar to Fama and French's (2001) idea that dividend payers tend to be high profitability, of great size and have weak investment opportunities; with non-payers being of the inverse nature. The `disappearing dividend phenomenon' is primarily present in firm choice, i.e. how they wish to purpose their cash., and; the fact that there is a great enough quantity of alike firms to decrease aggregate dividend value, even though many dividend payers are constantly increasing the number of dividends they pay and their value.

	\newpage

	\subsection{Question 2: Payout Policy \& Agency Theory}

	\textit{Describe the agency conflicts that may exist between shareholders and managers in the context of corporate payout policy. Using appropriate theory, critically evaluate how payout policy could reduce this agency cost. Your evaluation should consider both the level of cash payout and the method of payout.}

		\subsubsection*{Background}

	Agency problems are much like mis-defined alias' in your .zshrc. Much like an alias is expected to act on behalf of another, perhaps longer and more complex, command; agents are expected to make small movements in their own operations which not only payoff in their respect but, also are triggered in the interests of a relationship with another party. Just as a mis-match of $<$command$>$, $<$operation$>$, $<$function$>$ and $<$program$>$ leads to unexpected results when executing alias', a firm's management's relationship with their shareholders can quite as easily be interrupted by any of the former.\\

	Primarily in this context, problems are spawned in firms where there is a `principal-agent' problem, in which there is a dispute between two parties: the principal and the agent. Problems often occur when decisions are made by agents which overwhelmingly do not act in the favour of the corresponding party, as mutually respected. If this happens, it is likely that the agent is making an attempt to act in self-interest or in some-sort of `punishment' of the principal. The odds are sometimes skewed in the favour of the agent, in the sense that they may have access to greater/more information; described as `asymmetric information'.

		\subsubsection*{Origin of Issues}

	Easterbrook's (1984) study suggests, in findings, that the interests of managers are rarely in a perfectly positive relationship with those of shareholders. Most commonly, it's found that problems originate in the context of payout policy. More specifically, when managers favour share re-purchases for example, over significantly increased dividends. Expanding upon this, Jensen (1986) makes the claim that firms with significantly greater cash flows always favour re-investment but, in the naïve sense where the rate of return is [less than] the cost of capital. Thus, resulting in a [negative] NPV project. This is often referred to as the 'free cash flow' problem.\\

	Furthermore, Knyazeva and John (2006) argue that many firms use payout policy again to attempt to dilute the effects of these issues. The issue with many firms is that there I a trade-off present between how issuing higher dividends and making repurchases look. Repurchases are argues to appear as a more impulsive decision, in a situation where a firm acquires an unexpected cash flow for example. Clearly in this case, it is not logical to issue higher dividends no matter how great the cash flow may be; due to the unpredictability and lack of stability in this context of cash acquisition. Repurchases are generally used as the quicker and more flexible decision in payout policy due to their tie to adjustability and market value. But clearly, they can anger naïve investors and may not solve agency problems present. Dividends are more 'secure', arguably. Investors trust them and are loyal to high-dividend payers. They are less sporadic and are used to offer consistency, relevant to a firm's earnings; rewarding shareholders proportionately and correctly. However, many investors may become greedy at the entrance of new cash and do not see the favourable corporate decisions; simply seeking higher payoff.

		\subsubsection*{Justification}

	Easterbrook (1984) suggests that a primary justification for a firm using [increasing/consistent] dividends is to solidify their role in a capital market; reducing their cash flow costs. This may be a cheaper, more consistent and well- respected choice of payout policy however, it carries connotations of low managerial effort, belief and growth-responsibility.\\

	Furthermore, Chae et al. (2009) makes the argument that firms have the ability to optimise their payout policy function; to the criteria of low-transaction-cost, consistent/increasing payoff and shareholder satisfaction. This again ties back to the idea of the cost of dividends vs. repurchases. Dividends are more costly to maintain/maintain rises in. Not only does it involve large transactions, it also holds potential for the requirement of other external financing in the future. For example, where a firm is in financial distress and a priority is to satisfy shareholders. Dividend payments hold high risk in this sense, where much of the freedom in the decision making process is removed from firms' capabilities. Equally, retaining cash, investing large sums (in risky/uncertain projects) and engaging in repurchases can lead to shareholder disbelief and the thought that they are having potential stripped from them. Chae et al. (2009) does not offer a specifically viable proportionate optimisation model for these situations however. They do however, suggest the linear analysis of the cost of agency issues (spawned by previously discussed causes) versus cost of making long-term dividend payments/increases, in the favour of shareholders. Conclusively the, perhaps na\"{i}ve, statement is made that stronger corporate governance regulation is able to give power to shareholders and allow them to have a stronger influence in relevant payout decisions. In effect, removing much of the ability of manager to make crucial firm operative-maximising decisions.\\

	Allen et al. (2000) introduce the 'agency model' which highlights consistently increasing dividends appears attractive to institutional investors, such as hedge funds etc., where there is a more reliable and consistent payout offered to their clients. In this case, if firms underperform respected promises, investor have the ability to display greater influence. For this reason, management often takes more precaution in the presence of institutional investors; ensuring loyalty and consistency to them. In support of this, Kumar's (1988) study highlights 'dividend smoothing', through which dividends are actually argued to not perfectly signal a firm's position. It is generally assumed in these circumstances that management has access to more accurate information regarding firm operations and optimisation, which investors do not. This influences investment decision making. For example, managers may attempt to influence investment by altering reports on firm productivity etc.\\

	To further consolidate on the issue of repurchases lacking desire in investors, Brennan and Thakor (1990) explore the idea that better-informed shareholders, regarding firm projections and plans, often take advantage of their additional knowledge in the case of a possible repurchase. They engage in bidding when stock is priced higher and do not engage when its priced lower. This is in aim of outperforming lesser-informed investors. More-informed investors like repurchases because of this however, lesser-informed ones will favour dividends as they are essentially guaranteed equally-as-good a payoff as all other investors. Conclusively, repurchases may be a smarter option for management in many cases however, associated agency costs and loyalty issues may cause trouble in the future. Dividends are mutually favourable in that sense.

	\newpage

	\subsection{Question 3: Catering Theory}

	\textit{Piss off.}\\

	Piss off.
	
\newpage

\section{Section 2: Capital Structure}

	\subsection{Question 1: Agency Theories of Capital Structure}

	\textit{Zweibel (1996) identifies numerous shortcomings associated with normative agency based models of capital structure. Explain these shortcomings and critically evaluate how they are addressed in Zweibel's managerialist model of capital structure. Your evaluation should make reference to empirical evidence on the determinants of capital structure choice.}\\

	Piss off.

	\newpage 

	\subsection{Question 2: Asymmetric Information}

	\textit{Explain the time-varying asymmetric information explanation for market timing behaviour of companies put forward by Korajczyk, Lucas and McDonald (1992), along with the empirical evidence in support of this model. What are the empirical shortcomings of this study?}

		\subsubsection*{Background}

	One of the most recognised propositions for the explanation of capital structure is asymmetric information. This issue is also referred to as 'information failure', which refers to a case in which one party, relating to a firm or other economic scenario, has greater/better information than another. In a case where information asymmetry is abused, for example when directors may have an information surplus over other shareholders, there is a case of adverse selection; ``sellers have more information than buyers''. Obtaining and utilising a surplus of information in search of abnormal self- interest and gain is established at the cost of other parties.\\

	There are two primary sources of theory in this context. Myers and Majluf (1984) explore a 'pecking order' theory in which they claim there is no optimal capital structure. More specifically, they state that when managers lack greater information than other associates, safer investments are preferred over riskier ones; due to their lack of intellectual leverage. Korajczyk et al. (1992) argue that this issue of asymmetric information is time-varying and has different effects in different periods.

		\subsubsection*{Time-Varying}

	Korajczyk et al. (1992) actually disagree with Myers and Majluf (1984) on the issue of time-variance in asymmetric information. The former presents the idea that it varies based on factors such as disclosures and announcements by directors, such as earnings. They believe this to be the primary source of 'information' when referencing asymmetry and, that its influenced by the frequency at which directors choose or are required to disclose it. In findings, Korajczyk et al. (1992) suggest that there is the highest quantifiable 'amount' of asymmetry pre-announcement but, there is a significantly lower 'amount' of asymmetry immediately following an announcement. Inferred from this is the outline that the variance is based on specific periods and patterns; asymmetry does not vary randomly over an unlimited time-period. Korajczyk et al. (1992) display a model in which they suggest 'temporal variation', which essentially just highlights different density/concentration interrelations of samples within a population; just like your iconic class-room titration experiment. The terminology is arguably over-kill for this context but fill your boots Korajczyk et al. It simply states that the concentration of announcements/disclosures is related to the degree to which asymmetric information is observed.\\

	Korajczyk et al. (1992) state [1] to deal with discussed implications, managers often concentrate equity issuance after information disclosures. They [2] assume that the ultimate price drop post-announcement should increase over time, relative to asymmetry. Also, Korajczyk et al. (1992) [3] suggest that managers who have access to greater information, specifically regarding their higher-values assets etc., actually postpone equity issuance until other parties are better- informed. Primarily as this stage is reliant on external investor trust and, ultimately, essential positive knowledge.\\

	Furthermore, the statement is made that more-certain projects arranged by management may be postponed as their timing is believed to carry no significant impact on their value. This is especially apparent in the case were firms create the option through these means to time equity issuance as the cost of waiting essentially zero. Korajczyk et al. (1992) suggest that when management does not have private information, they will exercise this option until they have access to private-information leverage. The above is primarily observed in better-quality- asset firms, where ones with lesser-quality assets follow a more unpredictable pattern; likely issuing equity at whatever time suits. It's argued that the delay would not have the same benefit to them, it may result in far higher risk in their projects' timelines. Generally, if projects are certain (usually more secure-asset firms) a delay could benefit; if projects are uncertain (usually lesser-asset firms) a delay may increase unnecessary risk.

		\subsubsection*{Value}

	Lucas and McDonald (1990) suggest that in fact, overvalued firms will always issue equity as soon as possible where, undervalued firms may delay issuance to account for price-correction. The academics highlight assumptions of this statement as follow: there are positive abnormal returns in periods leading to new equity issuance, suggesting the opportunity for information asymmetry is short due to information acquisition by externals post-issue. Further, they believe the number of issuances to be time-varying and echo general trends in the market. Lucas and McDonald (1990) find this to hold in the sense that, the trend of asset quality generally aligns with the trends in higher numbers of equity issuance. Also, they discuss the presence of negative abnormal returns on-and-around issuance announcement; leading to the idea that many investors may perceive this as an indicator of over-valuation in the firm.

	\newpage

	\subsection{Question 3: Market Timing Theories}

	\textit{Baker and Wurgler (2002) present empirical evidence suggesting that a company's capital structure is the cumulative outcome of past attempts to time equity markets. Explain this study, and critically analyse it with specific reference to related studies that support, or conflict with, their findings.}\\

	Piss off.

\newpage

\section{Section 3: Capital Markets}

	\subsection{Question 1: Determinants of Corporate Cash Holdings}

	\textit{Using empirical evidence, evaluate the influencing factors that determine the balance of cash and cash equivalents held by companies, and put forward explanations as to why the level of cash held by firms has changed over time.}

		\subsubsection*{Background}

	Many firms may wish to retain cash, as opposed to distributing it or investing it, for various reasons. These include time-varying desires for cash, precautionary movements, tax advantages and, self-interest.\\

	[1] If a firm ever instantaneously finds itself needing cash but they have a low/no cash reserve, there are far greater costs in raising the sums they need. For example, they may have to engage in lengthy liquidation processes. This does not satisfy the potential needs for instant access to cash to fund +NPV projects. These opportunities do not always wait for cash to be raised through external methods etc. This creates a trade-off between retaining cash for potential payoff in future projects and, investing/distributing extra cash using currently practices methods.\\

	[2] Of course, many firms practice cash retention as a risk reduction method in the sense that retaining some cash to use in financial distress is easier and safer than, for example, taking on debt to save themselves. This also involves holding a potentially 'useless' pile of cash for a period of time however, and creates another loss of potential earnings.\\

	[3] Firms use cash holding methods to avoid paying excessive amounts of tax. For example, firms operating over-seas are likely to frequently hold cash in foreign countries where it's earned to avoid paying taxes associated with exchange and domestic accounting/investing etc.\\

	[4] The final reason involves the more self-interest-based idea that managers may wish to utilise cash for personal needs when there are no apparent good investment opportunities. This selfishness is spawned through bias towards self-interest over desires of shareholders, e.g. payout.

		\subsubsection*{Transaction Motive}

	Baumol (1952) highlights the key factors in cash holding within a basic model which displays the 'optimal' amount of cash for a firm to hold at a given time. This follows:

	$$C^*=\frac{\sqrt{2T(TC)}}{r}$$

	Where:\\
	$C^*$ = Optimal Cash Holding\\
	$T$ = Total New Cash Required for Period\\
	$TC$ = Fixed Transaction Cost of Selling Securities to Raise Cash\\
	$r$ = Opportunity Cost of Holding Cash (Interest Rate)\\

	This model is basic and simply states that cash will be replenished in similar intervals over time. However, Miller and Orr (1996) build upon the model, adding upper and lower limits which define parameters at which a firm will either sell investments which in need of cash (lower limit $L$) or, make investments when cash is too high (upper limit $U$). They will ideally aim to fluctuate with a consistent std.dev around their optimal cash balance $C$. This model is further extended to 'optimise' daily cash holding and include more realistic parameters. This follows:

	$$Z^*=\sqrt{\frac{3(TC)(\sigma^2)}{4r}}+L$$

	Where:\\
	$U^*$ = Optimal Upper Cash Balance = $3Z^*-2L$\\
	$U$ = Upper Cash Limit\\
	$L$ = Lower Cash Limit\\
	$\sigma^2$ = Variance of CFs\\
	$r=\left(\sqrt[365]{EAR+1}\right)$

	$$\therefore\mathrm{Avg.\ Cash}=\frac{4Z-L}{3}$$

		\subsubsection*{Further Transaction Motive \& Precaution}

	Opler et al. (1999) however, argues that many upper limits defined by firms are too high and that they are simply using these levels to look wealthier to perhaps naïve investors. Thus, they ask the questions: is there actually an optimal level of cash on the balance sheet? And, is a large cash holding actually justified? They argue that greater amounts of cash should always lead to higher investment, even when firms are younger, have less cash and lower-quality investment opportunities. It is gathered that generally, cash decreases with larger firm sizes due to their ease of access to cheaper debt financing etc. and better investment opportunities. It also decreases with higher-liquidity firms as more liquidity can generally take on the role cash holding plays in certain circumstances. Higher- leverage firms hold less cash also in-thought of their interest payments to lenders. Payouts are generally found to replace cash holding. And, the more regulation a firm is restricted under, the less cash it holds; likely due to fees, taxes, restricted opportunities etc.\\

	Furthermore, cash tends to be higher in growth-oriented, even just investment- oriented, firms as even though they make many investments (R\&D etc.), they require a suitable reserve to be able to fund innovations and expansions at any time, for example. Additionally, firms with higher cash volatility generally tend to hold more cash, simply to account for large impulse decreases etc. A similar rule applies for industry volatility in that, firms in risky industries tend to hold more cash to escape any more 'macro' issues.

		\subsubsection*{Tax Motive}

	Many firms prefer to hold large sums of cash in foreign countries or make additional foreign investments to avoid the tax cost of exchanging with domestic currency. Fritz et al. (2007) find that generally, a one std.dev increase in tax cost results in a \~{}7.9\% increase in cash. The same study finds that firms defined as having a 'high' tax burden hold \~{}47\% of their cash in foreign locations; firms defined as having a 'lower' tax burden only hold \~{}26\% in foreign countries. They also find that firms with higher domestic leverage are less likely to delay taxes of returning foreign cash to domestic holdings.

		\subsubsection*{General Findings}

	Bates et al. (2009) found that cash levels generally have increased since the 1980's (to 2006); at a rough rate of 0.46\%yr$^{-1}$. Furthermore, cash ratios of firm have generally increased by \~{}10.5\%yr$^{-1}$ over the same period. This is arguably related to the idea of 'disappearing dividends' where; in-and-around the 1980's, dividends payers and non-payers were found to have a roughly equivalent average cash ratio. However in 2006, non-payers' cash ratio had significantly increased, where that of payers had seen little to no trend. Primarily, it is evident that increasing cash ratios are influenced by exiting-investors, higher CF risk, lower capital expenditure and higher impulsive/innovative/expansive investment/R\&D.\\

	Duchin (2010) extends research to precautionary diversification; stating that firms which operate in more than one industry generally look to hold lower sums of cash, primarily due to a safer spread of operations, investments and assets. They suggest \~{}11.9\% of a 'diversified' firm's assets are cash where; firms operating in one industry hold \~{}20.9\% in cash. These firms tend to take more 'precautionary' measures as they hold more condensed risk. This linearly translates to 'diversified' firms having 'more diverse' CFs and 'non-diverse' firms having 'less diverse' CFs.\\

	Finally, Acharya at al. (2007) suggests that firms may use higher cash holdings as an attractive balance sheet factor. For example, a firm with a cash value greater than its debt value is considered to have 'negative debt', which can be attractive to naïve investors or extremely debt-averse investors.

		\subsubsection*{Conclusions}

	It's summarised therefore that, throughout different requirements for time- varying desires for cash, precautionary movements, tax advantages and, self- interest; many firms adopt precise measures in order to ensure optimisation of cash levels for given periods in their lifetimes.\\

	Optimally, firms should seek a method under which they can maximise investment using leftover cash and reserve the minimum but essential amount for any form of calculated financial distress, to avoid debt issues. Smaller, younger firms ideally should invest if they wish to develop at a competitive rate; ``getting there first''. Larger firms tend to prefer debt for investment as it's cheaper with scale; sometimes targeting larger sums of retained cash only at distress. Generally, higher-taxed firms hold a greater amount of cash and, aim to reduce tax burden using foreign accounts.\\

	Financially constrained firms with high-hedging desires tend to allocate cash to cash holdings and, constrained firms with low-hedging desires tend to allocate more cash to debt and interest payments. But, unconstrained firms frequently use excess cash to reduce and 'hold-down' debt.\\

	Increasing cash levels, since the 1980's and throughout the 'disappearing dividend phenomenon', are evident in cases of exiting-investors, higher CF risk, lower capital expenditure and higher innovation and intellectual expansion.

	\newpage

	\subsection{Question 2: Issuing Equity}

	\textit{Describe the various motives that lead companies to undertake an initial public offering (IPO). Critically evaluate the relative importance of each motive in the context of existing theoretical and empirical literature.}

		\subsubsection*{Background}

	A large decision in a firm's lifespan is the one to go public. An initial public offering (IPO) is typically utilised when a firm is under-priced, which is primarily used to attract new external investors. This process is also frequently designed to reward early- coming investors with loyalty. The 'money left on the table' idea associated with this process describes the lost potential earnings in the process of engaging in an IPO. Ritter and Welch (2002) describe motivations for engaging in IPOs non-exhaustively as the following: access to capital markets, an exit strategy for initial investors/cashing-out, dispersal of ownership etc. Ritter (2015) argues that there is a large transfer of wealth from existing shareholders to IPO investors however, it can be costly. For example, Visa's IPO left \~{}\$5bn 'on the table' as IPO investors are sold shares at the offer price; the gap between offer and closing price after the first post- IPO closing day left this sum present.

		\subsubsection*{Capital Market Access}

	Firms tend to engage in IPOs when they seek to diversify and expand financing methods, highlight investment opportunities to possible investors and, attempt to reduce costs of raising capital from other providers (Ritter, Welch, 2002).

		\subsubsection*{Exit Strategy}

	Many entrepreneurs either wish to engage in business to grow firm and even an industry, not caring much about profits or other financials or (The Zuck, being a good example); they are in the business to grow their financials to the top than leave. Either way, they may want to exit [1] as they have arrived at where they wished to bring the industry to and see other experts as the people to carry on operations, doing as they wish or; [2] because they have the money they want. Zingales (1995) argues that IPOs are a good point at which for an entrepreneur to make their exit as it opens up opportunity for a higher-payoff. For example, giving exposure to acquirers. Alternatively, if entrepreneurs wish to regain control, the same applies in the context of venture capitalists exiting.

		\subsubsection*{Dispersal of Ownership}

	Chemmanus and Fulghieri (1999) state that older firms will generally prefer the logistics of IPOs over younger firms as they have a greater amount of cash to fund distribution of information etc. It is therefore easier for them to favour IPO investors over venture capitalists etc. Younger firms on the other hand, tend to benefit more from venture capitalists to get them off their feet, due to the lower costs associated with initiation in the context of venture capital.

		\subsubsection*{Competition}

	Maksimoc and Picher (2001) argue that seeing higher-valued firms which are listed tends to encourage firms' competitors to also engage in IPOs. This holds the potential to increase market competition, simultaneously increasing the hight at which the relevant competition is taking place. In effect, increasing the price of the firms, as more confidence and exposure is observed.

		\subsubsection*{IPO Empirical Studies}

	Pagano et al. (1998) investigate the effects of an IPO decision and make inference between the theoretical determinants of an IPO and a firms proposed probability of listing. They suggest that the probability of a firm engaging in a IPO is positively related to the size of the firm, the firm's profitability and book-to-market ratio. They find evidence that historically, profitability decreases post-IPO. They consider the fact that entrepreneurs time IPOs to align with high profitability, known as 'windows dressing'; temporarily making a firm look more attractive to investors who may be looking for a quicker investment. It therefore sometimes may come as a surprise when the firm's profitability decreases after the IPO. A lot of this decrease however, is as implied; manufactured by the timing of the entrepreneur.\\

	Furthermore, this study finds evidence that corporate investment and leveraging often decreases post-IPO. It is possible that many firms engage in an IPO to rebalance their capital structure and therefore, given a successful IPO, are left with less inclination or need to engage in previously-used cash-raising methods like debt financing.\\

	Finally, the same study finds that a firm's cost of credit decreases post-IPO. To extend the prior argument, a firm with decreased leverage is considered 'safer' [to lend to] and therefore may be offered better rates. Furthermore, with the increased volume of information required to be made available about a firm, in the case of being public, aids banks' information gathering process when considering lending and therefore translates to reduced fees charged to the firm. More information may also help banks realise 'credit-worthy' firms and therefore make their acceptance process more efficient.\\

	Lowry (2003) explores possible explanations for the volume of IPOs at given times. They suggest three primary explanations for IPO volume: [1] the business cycle, where economic expansions tend to free-up capital so there is a deeper pool of investors for IPOs engagers to offer to; [2] investor behaviour must be considered, in that investors may be bullish in reaction to current market conditions etc. so they'll tend to invest in anything new on the market but when they're bearish, it's difficult for an IPO'ing firm to attract investors. And, [3] information may cause deviations in opinion regarding the quality of IPOs, i.e. why firms are engaging in them. This holds potential to both attract or repel investors based on their opinions.\\

	Overall, Yung et al. (2008) segment IPOs into categories from 'hot' to 'cold' which imply IPO uncertainty. The 'hot' category contains firms which have greater IPO uncertainty thus, a greater amount of under-priced, greater dispersion in stock prices, and a higher volume of delisting by 'lower-quality' firms.

	\newpage

	\subsection{Question 3: Choice of Borrowing}

	\textit{Publicly traded companies typically have access to various sources of debt finance. Using appropriate theory, discuss the various considerations of source and maturity structure of debt financing that a firm may undertake.}

		\subsubsection*{Background}

	When considering how publicly traded firms are able to finance operations, there are two primary categories which can be explored. These include public debt markets, which includes the issuance of public bonds and; private debt markets, which involves utilising bank loans, hedge funds, other funds, other lenders etc. The clear, traditional 'lender-to-borrower' option here is private debt.

		\subsubsection*{Borrowing Methods}

	When exploring the options within private debt, firms have the ability to consider bilateral loans and syndicated loans. Bilateral loans are issued in a manner where one other firm, such as a bank/fund etc., provides cash to the borrowing firm. This is the most linear path taken in the debt market, straight from lender to borrower. It is most likely that if a firm engages in this type of debt financing, they wish to maintain in-house knowledge and oversight of their financing operations; usually looking to fund shorter-term operations such as working capital and/or capital expenditures such as more necessary purchases at a close date, like equipment, maintenance, upgrades etc.\\

	Alternatively, syndicated loans are a more diverse way of raising capital. Through this method, multiple lenders join to loan a firm cash, with the total sum being divided across these lenders. The terms of loans usually remain roughly the same as they would with just one lender. Furthermore in this case, there is an intermediary called an 'agent' who manages the logistics and legalities of the loan on behalf of the firms. The formation of a 'syndicate' of firms/intermediaries in this case often leads to a more efficient workflow when organising and executing the loan process. The borrowing firm is therefore able to repurpose time and effort efficiently, possibly engaging in operations associated with the reason for borrowing, at a faster rate. Typically, this method can be used for longer-term projects such as mergers and acquisitions, buyouts, larger and longer-term logistical purchases etc. Intuitively, this method benefits lenders as they are only risk-exposed to their portion of the loan, with the agent holding a portion of legal risk and responsibility also – benefitting both the lenders and borrowers. This method allows a borrower to work towards a better reputation as they get to prove their loyalties over the smaller, risk-divided syndicate.\\

	Furthermore, firms have the option of using others such as hedge funds, and other high-value funds alike. This is not a preferable option however, due to the significantly higher cost of borrowing in fees, interest rates etc. This is likely used if a firm needs a significantly high sum of cash, where the additional costs are insignificant compared to the main transaction itself.

		\subsubsection*{Theories of Debt Financing}

	There are three categories of theoretical background which some academics find to be present when observing lending. These include: information costs, efficiency of liquidation and agency costs and reputation.

		\subsubsection*{Information Costs}

	Fama (1985) argues that public markets often require access to a greater amount of information disclosure before issuing a loan to a firm. To many firms who often use high-volume debt financing methods, the cost of producing this information is insignificant compared to their worth. However, smaller firms may not have the cash or man-power required to produce this information in-time or at all. Therefore, smaller firms tend to lean towards bank loans who require less information or perhaps even information services which are designed to help acquire a minimum necessary folio of information, at a lower cost. Clearly, this all means larger firms have greater/easier access to public debt markets due to their ability to produce and disclose information, especially in circumstances where it could provide other benefits.\\

	Yosha (1985) also highlights the idea that revealing necessary information, perhaps for the first time, in search of a public loan puts firms at risk of revealing sensitive information to rivals. This is especially important in smaller firms and young firms where they may not have acquired/established all of their regulatory or legal parameters yet; such as trademarks, patents, acquisitions and purchases etc. This leads to the conclusion that firms with highly-valued projects/operations tend to lean towards private debt to avoid the above and, firms with lower-valued or, in this case, 'less-secret' projects/operations don't mind engaging in the public debt market.

		\subsubsection*{Efficiency of Liquidation}

	In debt issuing, agreements/promises, 'covenants', are made between the lender and borrower. Affirmative covenants provide parameters under which a borrow is required to perform/engage in specific actions; examples listed non-exhaustively: complying with the relevant laws and bylaws etc., maintaining accurate financial statements, maintaining appropriate insurance. Violations made by the borrower result in an immediate default. Though, a clause may exist in the contract allowing for time to amend compliance issues. At the other end, negative covenants exist to prohibit the borrower from engaging in particular activities which could significantly negatively impact the integrity of their credit and ability to repay debt. Most closely analysed, with regards to the associated criteria set out by the lender, are a set of financial ratios within the borrower. Analysis of these ratios is focussed on factors such as other debt, equity and earnings which imply ability to pay interest/repay debt sum etc. Again, a violation would likely result in a default and downgrading of credit rating of the borrower.\\

	Berlin and Mester (1992) describe a decision within firms seeking debt, between monitored contracts from banks which are better-regulated, with the ability of banks to examine corporate information about firms which isn't available publicly and; unmonitored contracts from public methods, which may have a greater deal of negative covenants due to the nature of lenders in this case. Therefore, firms are left with a monetary trade-off between the higher cost of bank monitoring – 'safer' and more 'desirably' regulated method and; inefficient and risky liquidation caused by harsher covenants in the public market.

		\subsubsection*{Agency Costs \& Reputation (Diamond's (1991) Model)}

	Diamond (1991) lays out a general model outlining the logistics of borrowing; including a firm seeking debt financing, an option to borrow in the public bond market and, an option to borrow directly, privately through an institution. They highlight an issue in private debt financing here, the 'moral hazard', in that a lender/borrow may gain incentive to deviate from the agreed path for personal gain. Most commonly, this involves activities which increase the risk of the borrower not being able to repay debt. For example, a firm may owe $X$ in interest but they only hold $X-k$ in cash; this means they essentially have 'nothing to lose' and are indifferent between making interest payments safely, or using remaining cash to fund risky projects which may payoff, allowing them to pay interest with some left over.\\

	The same model displays an outline of how the reputation of firms can influence borrowing. Typically, higher-reputation firms have freer access to the public market as they are trusted amongst the public and they have the financial and intellectual ability to share necessary information. Diamond (1991) argues that firms at this end of the spectrum will have incentive to maintain higher reputations and thus, reduce the risk of moral hazards. Otherwise, the borrowing process, and subsequent use of the cash, would be inefficient. In the centre of the spectrum are firms with a lightly blurrier reputation. They tend to borrow from banks as their rating may be lower. However at this level of observation, the monitoring factors is still argued to remove the moral hazard. On the other end, the lowest-reputation firms are argued to borrow primarily from non-bank private institutions. These firms have less reputation to lose if they default and therefore, don't mind entering the repetitive cycle of doing so. They also may disclose poor information but again, it's not as important in this case.\\

	Diamond (1991) makes the comprehensive statement that [1] firms borrow and make interest payments with desire to gain a good reputation; [2] the higher the reputation, the cheaper the source of debt and related costs/fees and; [3] a hierarchy from firms borrowing in the public market, to the non-bank private market is established. On-a- whole, models have predicted that preference for public debt is positively related to firm size, reputation and monitoring costs. Its negatively related to project quality and information access.

		\subsubsection*{Analysing Diamond's (1991) Model}

	Denis and Mihov (2003) use the 'Altman Z-Score' as a metric for the financial stability of a firm. This metric is argued to predict the probability of bankruptcy using the following variables: (working capital/total assets), (retained earnings/total), (EBIT/total assets), (market value of equity/book value of liabilities), (net sales/total assets). It divides results into three 'zones': 'safe', 'grey' and 'distress'. They are self- explanatory. Subsequently, these scores align with bond ratings from AAA to D etc. Denis and Mihov's (2003) findings were aligned with theories of Diamond's (1991) model in that non-bank private borrowers have the highest probability of bankruptcy, 0.208 (20.8\%). However, they don't find much room to distinguish between private bank borrowers (0.121) and public borrowers (0.134). Primarily, the study concludes that it's likely the non-bank private borrowing market will be populated by low profitability, low credit-rating, low Z-Score firms. 75\% of public borrowers have obtained a rating of BBB or more; only 7\% of non-bank private borrowers have (Denis, Mihov's, 2003). Finally, it's found that non-bank private borrowers tend to be smaller, less profitable and have poorer interest cover.\\

	In-line with this observation, Hadlock and James (2002) investigate firm characteristics further within the debt market; finding that public debt is indeed typically used more by larger, older firms. Likely with a greater reputation. This further aligns with Diamond's model. In empirical study, they find overvalued firms utilise the public market and thus, when firms are undervalued they go towards the private market; seeing negative returns leading to the announcement of debt, with positive return on the day (Hadlock, James, 2002). Thus concluding that many user s of bank debt are smaller and have a greater return volatility – as a measure of risk. Furthermore, Billet et al. (2006), find in many cases that firms see underperformance post-bank loan; with abnormal returns as great as ($-$)33\%. However, this was primarily the case amongst larger firms.\\

	Furthermore, in-line with the idea of reputation, Datta et al. (1999) suggests a firm entering the public debt market for the first time must already have an established reputation with banks; meaning associated costs, such as monitoring, are likely to be reduced due to loyalty. In result, finding a decrease in yield-spread with increasing age and relationship status with the relative bank. Thus aligning with Diamond's (1991) idea that relationship stability decreases cost of debt. Thus, a firm incentive to improve reputation and act in favour of both themselves and the bank is established.

\end{document}
